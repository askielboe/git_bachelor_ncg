\section{Results}
In the following section we will present our results, and numerical considerations, relevant for overviewing the possible consequences of NCG at modern collider experiments. We will give an estimate on the possible scale of a NCG theory, and what impact this will have on cross sections for processes involving the $Z$ boson.

\subsection{Setting constraints on $\Lambda$ using LEP data}
To set a constraint on the possible values of $\theta = \frac{1}{\Lambda^2}$ we can look at the width of $\Gamma_{Z \rightarrow gg}$ as predicted by NCG. Now, we have not seen effects of NCG greater than the uncertainty in current experimental data, additional contribution to the total $Z^0$ width, coming from $Z \rightarrow gg$ processes, must thus be lower than the current uncertainty $\Gamma_{Z \rightarrow gg} < 1 \times 10^{-3}$ GeV \cite{behr2003dnc}. To see how this puts a constraint on $\theta$ we use an expression for $\Gamma_{Z \rightarrow gg}$ taken from \cite{behr2003dnc};
\begin{equation} \label{eq:zggwidth}
	\Gamma_{Z \rightarrow gg} = \frac{8}{12} K_{gg}^2 \alpha M_Z^5 \sin^22\theta_W \frac{1}{\lambda^4}.
\end{equation}
In figure \ref{fig:kplot} we have made a plot of the minimum allowed value for $\Lambda$. Allowed values of $K_{gg}$ lies in the interval $\{-0.1,0.2\}$ \cite{behr2003dnc}, so our maximum lower bound for $\Lambda$ is found for $K_{gg}=0.2$. This gives the experimental minimal value $\Lambda >130$ GeV.
\includefigure{fig:kplot}{0.3}{./images/kplot}{Plot of the allowed region for $\Lambda$ and $K_{gg}$. The filled area represent the constraint $\Gamma_{Z \rightarrow gg} < 0.001 \textrm{ GeV}$ as set by equation \eqref{eq:zggwidth}. Allowed $K_{gg}$ is in the region $\{-0.1,0.2\}$ represented by the black box in the figure.}

\subsection{Cross sections for different $\Lambda$ compared to SM}

% \includefigure{fig:intcrosssection}{0.6}{./images/intcrosssection}{Just a little plot so it looks more finished.}