\section{Results}
In the following section we will present our results, and numerical considerations, relevant for overviewing the possible consequences of NCG at modern collider experiments. We will give an estimate on the possible scale of a NCG theory, and what impact this will have on cross sections for processes involving the $Z$ boson.

\subsection{Setting constraints on $\Lambda$}
To set a constraint on the possible values of $\theta = \frac{1}{\lambda^2}$ we can look at the width of $\Gamma_{Z \rightarrow gg}$ as predicted by NCG. Now, we have not seen effects of NCG greater than the uncertainty in current experimental data, additional contribution to the total $Z^0$ width, coming from $Z \rightarrow gg$ processes, must thus be lower than the current uncertainty $\Gamma_{Z \rightarrow gg} < 0.0023 \textrm{ GeV}$.\footnote{This value is taken from the 2008 edition of the Particle Physics Booklet by PDG \cite{amsler2008rpp}.} To see how this puts a constraint on $\theta$ we use an expression for $\Gamma_{Z \rightarrow gg}$ taken from \cite{behr2003dnc};
\begin{equation} \label{eq:zggwidth}
	\Gamma_{Z \rightarrow gg} = \frac{8}{12} K_{gg}^2 \alpha M_Z^5 \sin^22\theta_W \frac{1}{\lambda^4}.
\end{equation}
In figure \ref{fig:lambda_kgg} we have made a plot of the allowed region for $\Lambda$ and $K_{gg}$.
\includefigure{fig:lambda_kgg}{0.3}{./images/lambda_kgg}{Plot of the allowed region for $\Lambda$ and $K_{gg}$. The filled area represent the constraint $\Gamma_{Z \rightarrow gg} < 0.0023 \textrm{ GeV}$ as set by equation \eqref{eq:zggwidth}. Notice the singularity for $K_{gg} \rightarrow 0$.}

- Ny Z cross-section
- Constraints på konstanter

\includefigure{fig:intcrosssection}{0.6}{./images/intcrosssection}{Just a little plot so it looks more finished.}