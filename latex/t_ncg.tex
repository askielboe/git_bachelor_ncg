\subsection{Noncommutative Geometry}
One of the candidates for solving the problem of getting gravity incorporated in the SM is NCG. It is a mathematical description of space-time that makes each point have a spatial extension, as opposed to GR where each point is "pointlike". In GFT the position operators commute, $[ \hat x^{\mu}, \hat x^{\nu}] = 0$, this means that you can swap the positions without any trouble. You can say that the space-time is flat or that the points are infinite exact. In NCG the points are not welldefined and the space-time seems to fluctuate. The idea that coordinates may not commute can be traced back to Heisenberg [CITATION]. Allain Connes [CITATION] was the first to formulate the mathematical describtion for NCG. In 19?? he introduced a non-zero commutator between the position operators. 
\begin{equation}
[ \hat x^{\mu}, \hat x^{\nu}] = i \theta^{\mu \nu}
\end{equation}
here 	$\theta$ is a constant, real, antisymmetric tensor. $\theta$ can be understood as a background field not affecting free particles but only their interactions Therefore the Noncommutative Standard Model will keep the original gauge group and particle content.


\subsubsection{The minimum length scale}
By introducing this non-zero tensor the uncertainty relation between the position operators turns out to be:

\begin{equation}
\Delta \hat x^{\mu} \Delta \hat x^{\nu} \ge \frac{\theta^{\mu \nu}}{2}
\end{equation}

We remember the analogue from QM where $\Delta x^{\mu} \Delta p^{\mu} \ge \frac{\hbar}{2}$,the more precise the position, the less precise the momentum, or you can say that if you have an accurate measure of the porsition of a particle, the unsertainty principle sets a lower limit, $\frac{\hbar}{2}$, to how small the disturbance on the momentum measurement can be. In the same way NCG defines a minimum lenght-scale. Here the unsertainty of two positions conflate, and creates a minimum detectable area which is dependent on the magnitude of $\theta^{\mu \nu}$.

\subsubsection{New interactions}
Among the features which are novel in comparison with the SM is the appearance of additional gauge 
boson interaction terms.
- Den nye kommutator fører til ny action som fører til nye feynmanregeler som fører til nye koblinger
- Z <-> gg koblingen og hvorfor den er interessant (i forhold til at teste teorien ved LHC)


