\subsection{Noncommutative Geometry}
One of the candidates for solving the problem of getting gravity incorporated in the SM is NCG. It is a mathematical description of space-time that makes each point have a spatial extension, as opposed to GR where each point is "pointlike". In GFT the position operators commute, $[ \hat x^{\mu}, \hat x^{\nu}] = 0$, this means that you can swap the positions without any trouble. You can say that the space-time is flat or that the points are infinite exact. In NCG the points are not well defined and the space-time seems to fluctuate. The idea that coordinates may not commute can be traced back to Heisenberg \cite{snyder1947qst} but Allain Connes \cite{connes1991pma} was the first to formulate the mathematical description for NCG. In 1990 he introduced a non-zero commutator between the position operators; \cite{rosendahl2008} 

\begin{equation} \label{eq:ncgtheta}
	[ \hat x^{\mu}, \hat x^{\nu}] = i \theta^{\mu \nu} = i \frac{1}{\Lambda_{\textrm{NCG}}^2} F^{\mu\nu},
\end{equation}
where $\theta^{\mu \nu}$ is a constant, real, antisymmetric tensor, $\Lambda_{\textrm{NCG}}$ is a scalar quantity which has the unit of energy and $F^{\mu\nu}$ is the antisymmetric field strength tensor from classical electrodynamics.\footnote{See section 11.9 in \cite{jackson1975cew} on covariance and special relativity in electrodynamics.} The commutator, $\theta^{\mu \nu}$, can be understood as a background field not affecting free particles but only their interactions. Therefore the non-commutative Standard Model will keep the original gauge group and particle content \cite{rosendahl2008}.
%We will later, when calculating  choose to ignore the space direction dependence induced by the field strength tensor and make the approximation $\theta = 1/\Lambda_{\textrm{NCG}}^2$, where $\Lambda_{\textrm{NCG}}$ is a scalar quantity which has the unit of energy.

\subsubsection{The minimum length scale}
By introducing this non-zero tensor the uncertainty relation between the position operators turns out to be:

\begin{equation}
\Delta \hat x^{\mu} \Delta \hat x^{\nu} \ge \frac{\theta^{\mu \nu}}{2}
\end{equation}

We remember the analogue from QM where $\Delta x^{\mu} \Delta p^{\mu} \ge \frac{\hbar}{2}$,the more precise the position, the less precise the momentum, or you can say that if you have an accurate measure of the position of a particle, the uncertainty principle sets a lower limit, $\frac{\hbar}{2}$, to how small the disturbance on the momentum measurement can be. In the same way NCG defines a minimum length scale. Here the uncertainty of two positions conflate, and creates a minimum detectable area which is dependent on the magnitude of $\theta^{\mu \nu}$.

\subsubsection{New interactions}
The inclusion of non-commutative geometry into the standard model adds new terms to the SM Lagrangian. Specifically we want to look at a term which allows the $Z^0$ boson to interact with gluons, the mediators of the strong interaction. This amplitude is strictly zero in the ordinary SM, but in NCG we get a term in the Lagrangian which looks like \cite{melic2005smn}
\begin{align} \label{eq:zggterm}
	\mathcal{L}_{Zgg} = &\frac{e}{4}\sin{2\theta_W}\textrm{K}_{Zgg}\theta^{\rho\sigma}
	[2 Z^{\mu\nu} (2 G_{\mu\rho}^a G_{\nu\sigma}^b - G_{\mu\nu}^a G_{\rho\sigma}^b) \nonumber \\
	 &+ 8 Z_{\mu\rho} G^{\mu\nu,a} G_{\nu\sigma}^b - Z_{\rho\sigma} G_{\mu\nu}^a G^{\mu\nu,b}
	] \delta^{ab},
\end{align}
where $K_{Zgg}$ is a dimensionless constant in the region $\{-0.1,0.2\}$ \cite{behr2003dnc}, $\theta_W$ is the Weinberg angle and $\theta^{\rho\sigma}$ is given in \eqref{eq:ncgtheta}. Using this term one can derive the corresponding feynman rules. Doing this one arrives at a non-zero matrix element for process including $Z^0gg$ vertices. \cite{melic2005smn} This is exactly the vertex we want to study. [MEN HVORFOR ER NETOP DET VERTEX INTERESSANT!?]