\section{Conclusion}
Hopefully LHC will start up this summer and within a few years reach the maximum energy of 14 TeV. The maximum energy currently available is at the Tevatron, and that is only running 2 TeV. The LHC is designed to reach energies of up to 14 TeV, this is a quantum leap in the power available to particle physicists, and may produce results that can one day revolutionize our way of understanding physics on the high energy scales. In either case, LHC may be able to give us a hint as to which direction we should work. In our bachelor project we have worked with Non-commutative geometry, one of the many proposed extensions to the SM. It have mostly been a literarily study of the, not so familiar, subject. We have builded, step by step, a fundament for understanding how QFT, GR, the SM, CompHEP and ROOT work together to produce predictions in HEP. Using these tools we have come to conclusions regarding the visibility of NCG at the LHC.

By using data from LEP we have been able to put a lower limit on the scale at which NCG could become visible. In the case of maximal interaction we have found this limit to be $\Lambda > 117$ GeV, see figure \ref{fig:kplot}.

Furthermore we have studied muon pair-production at the LHC using numerical tools to make several histograms of the total cross section including NCG contributions. Compared to what we expect from ordinary SM calculations the NCG cross section have been shown to become visible above the statistical uncertainty for $\Lambda < 500$, see Figure \ref{fig:lambdaplot}. Given that we do not detect any anomaly in the cross section for pair production when LHC probes the given energies, we can put a lower limit on $\Lambda$ of the order $\Lambda > 1000$ GeV. This is compatible with studies previously made by P. L. Rosendahl \cite{rosendahl2008}, who put a limit $\Lambda_{Rosen} > 2319 GeV$.

If we should actually detect an increase in cross section comparable to what we have seen here, NCG could be one of the explanations that should be look into.

[MAN VIL MAN IKKE KUNNE SE EN SYSTEMATISK FEJL HVIS NCG KURVEN LIGGER LIGE PÅ GRÆNSEN!?]

%What if we actually see a much larger Z cross section at high energy levels? Can we then confirm NCG or could there be other (new) explanations?

%This is not always an easy job. Actually we are in the middle in every sense. In our education, in a course about experimental high energy physics and in the proceeding with LHC at CERN. As we every week gain more knowledge, our level of understanding changes. This makes it difficult to feel like having an overview and sometimes we feel a sudden urge to add all our updated informations.\\