\begin{figure}[!htb]
\begin{center}
	
	\begin{fmffile}{one} 	%one.mf will be created for this feynman diagram  
	  \fmfframe(1,7)(1,7){ 	%Sets dimension of Diagram
	   \begin{fmfgraph*}(110,62) %Sets size of Diagram
	    \fmfleft{i1,i2}	%Sets there to be 2 sources 
	    \fmfright{o1,o2}    %Sets there to be 2  outputs
	    \fmflabel{$e^-$}{i1} %Labels one of the left sources
	    \fmflabel{$e^+$}{i2} %Labels one of the left sources
	    \fmflabel{${\ensuremath{\erlpm}}$}{o1} %Labels one of the right outputs
	    \fmflabel{${\ensuremath{\erlpm}}$}{o2} %Labels one of the right outputs
	    \fmf{fermion}{i1,v1,i2} %Connects the sources with a vertex.
	    \fmf{fermion}{o1,v2,o2} %Connects the outputs with a vertex.
	    \fmf{photon,label=$\gamma/Z^0$}{v1,v2} %Labels the conneting line.
	   \end{fmfgraph*}
	  }
	\end{fmffile}
	&&&&
	
	\caption{S-Channel left, T-Channel right}\label{fey1}
	\end{center}
	\end{figure}
