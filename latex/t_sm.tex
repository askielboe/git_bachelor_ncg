\section{Theory}
\subsection{Quantum field theory}
The particles and interactions of the SM are described in the language of quantum mechanics and quantum field theory.

\subsubsection{Principle of least action}
The action $\mathcal{S}$ is a functional\footnote{A functional is something that takes as its argument a function and returns a scalar.} related to the time evolution of a physical system. Namely the action is the change in phase of the wave function of the system. The Lagrangian is defined as the rate of change of the phase and given as

\begin{equation}
	\mathcal{S} = \int_{t_i \mathcal{P}}^{t_f} L \textrm{d}t,
\end{equation}

where we integrate over all intermediate states taken by the system evolving from time $t_i$ to $t_f$. The principle of least action then states that the actual path taken by the system is one that minimizes the action, or more precicely one where the action is stationary. In other words the variation of the line integral going from $t_i$ to $t_f$ is zero \cite{goldstein1959};

\begin{equation}
	\delta \mathcal{S} = \delta \int_{t_i}^{t_f} L \textrm{d}t = 0.
\end{equation}

Using Hamilton's principle, and knowing the Lagrangian of the system, one can derive the equations of motion.

\subsubsection{The path integral formalism}
In 1948 Richard Feynman presented his new formulation of non relativistic quantum mechanics, using what is known as the path integral formalism \cite{feynman1948sta}. The path integral takes into account all possible paths that a particle, or any other quantum mechanical system, going from a state $\ket{i}$ to a state $\ket{f}$ can take. This can be changes in position, momentum, energy or other intrinsic variables and is represented as an intermediate state of the system. Generally, and this is actually a central point in quantum mechanics, we are not interested in the path taken, but only in the fact that at some time $t_{i}$ the particle was in a state $\ket{i}$ and at some later time $t_{f}$ the particle is in a state $\ket{f}$. In order to arrive at this quantum probability-amplitude we must sum, or integrate, over the infinitely many paths, that the system can take to go from initial to final state. Doing this the amplitude is given as  \cite{richter_path_integrals}

\begin{equation} \label{eq:path}
	\bra{f} U(t_f,t_i) \ket{i} = \mathcal{N} \int \mathcal{D}x e^{i\mathcal{S}},
\end{equation}

where $\mathcal{N}$ is a constant of the system, $U$ is the time translation operator, $\mathcal{D}$ is an abbreviation given by the relation 

\begin{equation}
	\int \mathcal{D} x \equiv \lim_{N \to \infty} \int dx_1 \dots dx_N
\end{equation}

and the action $\mathcal{S}$ is related to the Lagrangian by the action principle, given in the 1-D case by

\begin{equation}
	\mathcal{S} = \int_{t_i}^{t_f} L(x,\dot{x},t)dt,
\end{equation}
where $x, \dot{x} \equiv \frac{dx}{dt}$ are generalized coordinates.

Equation \eqref{eq:path} is called the propagator in QFT. This equation describes the probability amplitude that a particle will go from the initial state to the final state in time $t_f - t_i$.

\subsubsection{Symmetry groups and gauge theories}
Noethe's Theorem states that for any transformation under which a physical system is invariant there exists a conserved quantity of that system. The classical examples are translations and rotations in the three spatial dimensions. If the system is invariant under those two transformations this corresponds to the conservation of linear and angular momentum. Thus these transformations forms a symmetry group of the theory. The two, physically equivalent situations, can be described by different mathematical configurations depending on the reference frame. The two descriptions are related to each other by this symmetry group.

In physics, any field theory in which the Lagrangian is invariant under some kind of continuous transformation is called a gauge theory, the invariance referred to as a gauge invariance. The group of transformations under which the Lagrangian is invariant is called the gauge group of the theory\footnote{For a historical account of the development of gauge theories, at a level relevant for the present discussion, see  \cite{gross1992gtp}.}.

\subsubsection{Quantum electrodynamics and SU(1)}
In quantum mechanics the wave equation is invariant under local, space-time dependent, phase transformations, corresponding to rotations in the complex phase space of the wave function. In practice this invariance comes about when, in order to make predictions of physical observable, we have to take the absolute square of the wave function. Doing this information about the phase of the complex wave function is lost and we are left with only the absolute size of the complex number. This means that we can never physically measure the phase of the wave function, only its amplitude. Because of this, if we want the wave functions to represent actual physical systems, we have to insist that our equations be invariant under all local phase transformations\footnote{Actually the argument is a bit more subtle. There is no physical principle that states that the wavefunction \emph{has} to be invariant under local phase transformations, it is just something we assume. And, as it turns out, we can learn a lot from this assumption. \cite{griffiths1987iep}} of $\psi$ given by

\begin{equation} \label{eq:localphase}
    \psi \rightarrow e^{i\alpha(\mathbf{x})} \psi.
\end{equation}

Together these transformations form the mathematical symmetry group SU(1). This is simply the group of all transformations of the form \eqref{eq:localphase}.  So far this does not seem to pose any problems for us. Adding a phase may change the wave function, but all predictions made about the position of the particle, described by this function, are left invariant. Specifically

\begin{equation}
	|\psi|^2 \rightarrow |e^{i\alpha(\mathbf{x})} \psi|^2 = |\psi|^2.
\end{equation}

However for the derivative operator $\partial_\mu$ to be well-defined for all points in space-time we have to replace it by the covariant derivative

\begin{equation} \label{eq:covariant}
	\partial_\mu \rightarrow D_\mu = \partial_\mu - \textrm{i}e A_\mu.
\end{equation}

The covariant derivative compensates for the fact that, when making the local phase change, we pick up an extra factor of $i (\partial_\mu \theta)$ in the derivative of the Lagrangian. So we have introduced a new vector field $A_\mu$, designed to exactly cancel this extra factor, making sure that gauge invariance is satisfied under local continuous SU(1) transformations. The field, it turns out, is actually just the electromagnetic field know from classical theory of electromagnetism. When $A_\mu$ is quantized in the context of quantum field theory the quanta of this field are the photons, know as the gauge bosons of QED. By insistence on local gauge invariance of the Lagrangian of the free electron we get the electromagnetic field, and corresponding interactions, almost for free. This principle is the driving force behind the success of gauge theories.

\subsubsection{Gauging the SU(2) and SU(3) Lie groups}
The previous discussion outlines many of the essential principles of gauge theories. Following this logic the gauge group governing weak interaction turns out to be the composed group SU(1) $\times$ SU(2), where SU(2) is the group of $2 \times 2$ complex unitary matrices\footnote{More specifically SU(2) is a Lie group and its Lie algebra is the set of anti-Hermition $2 \times 2$ matrices with trace 0. For more information about Lie groups and Lie algebras see  \cite{fulton1991rtf}.}. Thus electromagnetism and the weak interaction responsible for radioactive decay are combined into one theory, the electroweak theory. The generators of SU(2) are the Pauli matrices

\begin{equation}
	\sigma_1 =
	\begin{pmatrix}
	0&1\\
	1&0
	\end{pmatrix},
	\sigma_2 = 
	\begin{pmatrix}
	0&-i\\
	i&0
	\end{pmatrix},
	\sigma_3 = 
	\begin{pmatrix}
	1&0\\
	0&-1
	\end{pmatrix}.
\end{equation}

The gauge boson associated with the weak force are the $W^+$, $W^-$ and the $Z^0$ bosons.

The gauge bosons mediating the strong interaction are characterized by the SU(3) gauge group. The generators of SU(3) are the 8 Gell-Mann matrices
\begin{align} \label{eq:gellmannmatrices}
	\lambda_1 &= \begin{pmatrix} 0 & 1 & 0 \\ 1 & 0 & 0 \\ 0 & 0 & 0 \end{pmatrix},
	\lambda_2 = \begin{pmatrix} 0 & -i & 0 \\ i & 0 & 0 \\ 0 & 0 & 0 \end{pmatrix},
	\lambda_3 = \begin{pmatrix} 1 & 0 & 0 \\ 0 & -1 & 0 \\ 0 & 0 & 0 \end{pmatrix},
	\lambda_4 = \begin{pmatrix} 0 & 0 & 1 \\ 0 & 0 & 0 \\ 1 & 0 & 0 \end{pmatrix}, \nonumber \\
	\lambda_5 &= \begin{pmatrix} 0 & 0 & -i \\ 0 & 0 & 0 \\ i & 0 & 0 \end{pmatrix},
	\lambda_6 = \begin{pmatrix} 0 & 0 & 0 \\ 0 & 0 & 1 \\ 0 & 1 & 0 \end{pmatrix},
	\lambda_7 = \begin{pmatrix} 0 & 0 & 0 \\ 0 & 0 & -i \\ 0 & i & 0 \end{pmatrix},
	\lambda_8 = \frac{1}{\sqrt{3}} \begin{pmatrix} 1 & 0 & 0 \\ 0 & 1 & 0 \\ 0 & 0 & -2 \end{pmatrix},
\end{align}

giving rise to 8 gauge bosons of the strong interaction know as gluons, each having a property called color. Because of this color feature the gauge theory of the strong interaction is called Quantum Chromodynamics (QCD). The reason for having 8 gluons is that we need 8 matrices to span the 'color' space laid out by the Gell-Mann matrices, while requiring that they have determinant 1. This is in fact the same reason why we need three Pauli matrices to span the group of SU(2) [KAN MAN SIGE DET, MATEMATISK?].

\subsection{The Standard Model}
Combining electroweak theory and quantum chromo dynamics using the group U(1) $\times$ SU(2) $\times$ SU(3) theorists are able to account for a great variety of elementary particles and interactions, and make profoundly accurate predictions. This theory is know as the Standard Model of particle physics. The Standard Model is a non abelian gauge theory. \footnote{Non abelian meaning that, save for U(1), the generators of the gauge groups are non-commuting.}

\subsubsection{Particles of the standard model}

% Billede med partikel-generationer fra Wikipedia.
\includefigure{fig:particle_generations}{0.3}{./images/particle_generations.eps}{The fundamental particles in the Standard Model. Fermions and bosons, listed in their respective generations. (Source: Wikimedia Commons).}

In the Standard Model of particle physics we divide the particles into two main groups; fermions and bosons. Fermions are defined as having half-integral spin and are described by Fermi-Dirac statistics, these include the leptons and quarks. Fermions constitutes all known matter and as such they are sometimes described as matter-particles. Bosons, on the other hand, have zero or integral spin and are described by Bose-Einstein statistics. Some of these, namely the gauge bosons, are responsible for the weak, strong and electromagnetic interactions. Therefore the gauge bosons are often called the force-carriers of their respective interactions. It may be appropriate to note that the formalism of quantum mechanics makes no clear distinction between the concepts of matter-particles and force-particles. Below is a table with all known quarks, leptons and gauge bosons. With the exception of neutrinos, which have to be detected in other ways, all of them have been seen in particle accelerator experiments.

\subsubsection{Feynman diagrams}
When calculating amplitudes for particle interactions in the SM it can be very instructive to make use of \emph{Feynman diagrams}. This approach was proposed by american physicist Richard P. Feynman in 1949 \cite{feynman1949sta}. A typical space-time interaction the production of two muons from the interaction between an up and an anti-up quark, is represented as a Feynman diagram in Figure \ref{fig:feyn:uu_a_mm}. Space is extending top-down in the diagram while time is increasing from left to right. Arrows pointing along the flow of time, to the right, are particles, while arrows pointing against the flow of time, to the left, are anti-particles.

\begin{figure}[htp]
\centering
	\begin{picture}(170,90) (25,-22)
  \SetWidth{1.0}
  \SetColor{Black}
	\Line[arrow,arrowpos=1,arrowlength=2,arrowwidth=1,arrowinset=0.2](5,0)(15,0)
	\Line[arrow,arrowpos=1,arrowlength=2,arrowwidth=1,arrowinset=0.2](5,0)(5,10)
	\Text(18,-2)[lb]{\tiny{\Black{$t$}}}
	\Text(3,14)[lb]{\tiny{\Black{$\mathbf{x}$}}}
  \Line[arrow,arrowpos=0.5,arrowlength=5,arrowwidth=2,arrowinset=0.2](40,46)(70,16)
  \Line[arrow,arrowpos=0.5,arrowlength=5,arrowwidth=2,arrowinset=0.2,flip](40,-14)(70,16)
  \Line[arrow,arrowpos=0.5,arrowlength=5,arrowwidth=2,arrowinset=0.2,flip](140,16)(170,46)
  \Line[arrow,arrowpos=0.5,arrowlength=5,arrowwidth=2,arrowinset=0.2](140,16)(170,-14)
  \Text(28,49)[lb]{\Large{\Black{$u$}}}
  \Text(28,-27)[lb]{\Large{\Black{$\bar u$}}}
  \Text(172,49)[lb]{\Large{\Black{$\mu^+$}}}
  \Text(172,-27)[lb]{\Large{\Black{$\mu^-$}}}
  \Text(100,-4)[lb]{\Large{\Black{$\gamma$}}}
  \Photon(70,16)(140,16){5.5}{4}
\end{picture}
\caption{Feynman diagram for the interaction between two quarks, the up and the anti-up, and a virtual photon to produce a muon and an anti-muon. Time is increasing from left to right in the diagram.} \label{fig:feyn:uu_a_mm}
\end{figure}

The wavy line in the middle is a so called \emph{virtual particle}, in this case a virtual photon. It is virtual in the sense that we never actually measure it without disturbing the process and thereby destroying it. We can only ever measure the initial-state (incoming) particles and the final-state (outgoing) particles. Another reason why these, intermediate particles, are called virtual is suggested by looking at Figure \ref{fig:feyn:ee_a}. In the center-of-mass (CM) frame of the electron and positron, the two have exactly equal and opposite momenta. Together they have zero momentum in the CM-frame. Therefore the two particles can never annihilate to create a single photon, which always have momentum, given that it always travels at the speed $c$.

\begin{figure}[htp]
\centering
	\begin{picture}(170,90) (25,-22)
  \SetWidth{1.0}
  \SetColor{Black}
	% \Line[arrow,arrowpos=1,arrowlength=2,arrowwidth=1,arrowinset=0.2](5,0)(15,0)
	% \Line[arrow,arrowpos=1,arrowlength=2,arrowwidth=1,arrowinset=0.2](5,0)(5,10)
	% \Text(18,-2)[lb]{\tiny{\Black{$t$}}}
	% \Text(3,14)[lb]{\tiny{\Black{$\mathbf{x}$}}}
  \Line[arrow,arrowpos=0.5,arrowlength=5,arrowwidth=2,arrowinset=0.2](40,46)(70,16)
  \Line[arrow,arrowpos=0.5,arrowlength=5,arrowwidth=2,arrowinset=0.2,flip](40,-14)(70,16)
  % \Line[arrow,arrowpos=0.5,arrowlength=5,arrowwidth=2,arrowinset=0.2,flip](140,16)(170,46)
  % \Line[arrow,arrowpos=0.5,arrowlength=5,arrowwidth=2,arrowinset=0.2](140,16)(170,-14)
  \Text(28,49)[lb]{\Large{\Black{$e^-$}}}
  \Text(28,-27)[lb]{\Large{\Black{$e^+$}}}
  \Text(100,-4)[lb]{\Large{\Black{$\gamma$}}}
  \Photon(70,16)(140,16){5.5}{4}
\end{picture}
\caption{Example of a virtual process in which an electron and a positron annihilates to crate a virtual photon seemingly violating conservation of momentum at the vertex. This process can never happen on its own, but it can form part of other diagrams in a similar way to that in Figure \ref{fig:feyn:uu_a_mm}.} \label{fig:feyn:ee_a}
\end{figure}

To uphold energy and momentum conservation at the vertex the virtual photon have to have an energy different from that of a free photon. In principle it can take on \emph{any} value. The virtual photon is said to be \emph{off mass shell} or simply \emph{off-shell}\footnote{See chapter 2 in \cite{griffiths1987iep} for more about virtual particles.}.

Using these diagrams Feynman were able to greatly simplify how we think about and calculate interactions in the SM. Having drawn a diagram as in Figure \ref{fig:feyn:uu_a_mm} we now want calculate the probability for this process to happen. Given the diagram one can write down the amplitude using so called \emph{Feynman rules} associated with the different parts of the diagram. We won't go in to detail with writing down the complete feynman rules for \ref{fig:feyn:uu_a_mm}\footnote{For an account on how to write down the Lagrangian from feynman rules see \cite{peskin1993iqf}.}, but we can learn a lot by looking at the vertices.

Each vertex represents an interaction. Depending on the nature of this interaction we assign a constant related to the coupling between the lines going in to this vertex. This constant is proportional to the probability for the process to happen. For the diagram in Figure \ref{fig:feyn:uu_a_mm} the force acting at the vertices is the electromagnetic force. The strength of this force is determined by the fine structure constant

\begin{equation}
	\alpha_{\textrm{\tiny{QED}}} = \frac{e^2}{4\pi\varepsilon_0\hbar c},
\end{equation}

where $e$ is the charge of the positron, $\varepsilon_0$ is the permittivity of free space, $c$ is the speed of light in vacuum and $\hbar$ is the reduced Planck constant. In natural units, setting $c = 1$ and $\hbar = 1$, and at low energies $\alpha_{\textrm{\tiny{QED}}}$ is approximately equal to

\begin{equation}
	 \alpha_{\textrm{\tiny{QED}}} \approx \frac{1}{137}.
\end{equation}

Constants like these are appropriately called coupling constants\footnote{At increasing energies these constants are no longer constant, but are observed to be running, therefore they are also called running coupling constants. At energies of around $10^{14}$ GeV the fundamental forces are thought to combine into one grand unified force with one coupling strength $\alpha_{\textrm{\tiny{GUT}}}$. GUT stands for \emph{Grand Unified Theory}.}. The coupling constant associated with the strong force we will label $\alpha{\textrm{\tiny{S}}}$.

\subsubsection{Perturbation theory and renormalization}
The process in Figure \ref{fig:feyn:uu_a_mm} is not the only way to produce a muon/anti-muon pair from a up/anti-up pair. Just take a look at Figure \ref{fig:feyn:uu_a_mm_2}. In this case the photon splits up into a electron/positron pair which again annihilates into a photon, that goes on as if nothing had happened. Because of Heisenberg's uncertainty relation we can never, in any thinkable experiment, decide which one of the two processes actually happened, so we have to take them both into consideration. To calculate the total probability amplitude for the process $u \bar u \rightarrow \mu \mu^+$ we then have to include diagrams of these higher order perturbations as well. In fact we can add as many electron/positron pairs as we like. In principle, if we want to know the exact probability, we would have to include infinitely many diagrams, or processes, in our calculation. What saves us is the fact that as we add more vertices the overall probability is multiplied by factors of $\alpha^2 << 1$ for each vertex, so the total amplitude is convergent and disaster is avoided by the use of perturbation theory.

\begin{figure}[htp]
\centering
	\begin{picture}(170,90) (30,-22)
  \SetWidth{1.0}
  \SetColor{Black}
  \Line[arrow,arrowpos=0.5,arrowlength=5,arrowwidth=2,arrowinset=0.2](40,46)(70,16)
  \Line[arrow,arrowpos=0.5,arrowlength=5,arrowwidth=2,arrowinset=0.2,flip](40,-14)(70,16)
  \Line[arrow,arrowpos=0.5,arrowlength=5,arrowwidth=2,arrowinset=0.2,flip](170,16)(200,46)
  \Line[arrow,arrowpos=0.5,arrowlength=5,arrowwidth=2,arrowinset=0.2](170,16)(200,-14)
  \Text(28,49)[lb]{\Large{\Black{$u$}}}
  \Text(28,-27)[lb]{\Large{\Black{$\bar u$}}}
  \Text(202,49)[lb]{\Large{\Black{$\mu^+$}}}
  \Text(202,-27)[lb]{\Large{\Black{$\mu^-$}}}
  \Text(85,-4)[lb]{\Large{\Black{$\gamma$}}}
  \Text(151,-4)[lb]{\Large{\Black{$\gamma$}}}
  \Text(118,36)[lb]{\Large{\Black{$e^-$}}}
  \Text(118,-12)[lb]{\Large{\Black{$e^+$}}}
  \Photon(70,16)(105,16){5.5}{2}
	\Arc[arrow,arrowpos=0.5,arrowlength=5,arrowwidth=2,arrowinset=0.2,clock](120,16)(15,-180,-360)
  \Arc[arrow,arrowpos=0.5,arrowlength=5,arrowwidth=2,arrowinset=0.2,clock](120,16)(15,-0,-180)
  \Photon(135,16)(170,16){5.5}{2}
\end{picture}
\caption{Higher order contribution to the process of Figure \ref{fig:feyn:uu_a_mm}. The circle in the middle represents production of a virtual electron/positron pair which again annihilates to produce a photon, giving rise to a net polarization of the vacuum.} \label{fig:feyn:uu_a_mm_2}
\end{figure}

But the loop type diagrams as the one in Figure \ref{fig:feyn:uu_a_mm_2} poses another, more fundamental problem. Just as we had to sum over all possible diagrams, we have to integrate over all possible momenta of the virtual electron/positron pair in the loop. This integral turns out to be horribly divergent, we have an infinite number of ways in which momentum can be divided between the pair. Again we are saved, but this time the argument is a lot more subtle. The infinities are removed by the technique of renormalization. Essentially what we do is that we normalize by an infinite normalization constant, effectively and exactly canceling the divergent loop integrals. This is somewhat an empirical motivation, as we know that the probability has to be finite (actually is has to be $\leq 1$), but also because we know that the energy of the free electron is finite. What we do is then to include all loop type vacuum oscillations in the description of the electron, and since we can only ever measure the total energy (including vacuum oscillations in the immediate vicinity of the electron), we simply renormalize the otherwise infinite result when calculating the electron's energy, to arrive at the measured value. The complete discussion of renormalization is naturally a bit more complicated. Actually it took 18 years after the Yang-Mills article  \cite{yang1954cis} before Dutch physicist Gerardus 't Hooft was able prove  \cite{thooft1972rar} that standard model is indeed renormalizable. Renormalizability is a fundamental criteria that any theory hoping to describe the physical observable world has to satisfy. We will come back to renormalization when talking about general relativity and quantum gravity especially because this is a prime motivator in the development of noncommutative geometry.
