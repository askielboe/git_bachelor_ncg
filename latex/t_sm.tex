\section{Theory}
\subsection{Quantum mechanics}
The particles and interactions of the SM are described in the language of Quantum Mechanics (QM) and Quantum Field Theory (QFT) %, which have a natural way of unifying the mathematical concepts in a concise way.

\subsubsection{Principle of least action}
The action $\mathcal{S}$ is a quantity realted to the time evolution of a physical system. Namly the action is the change in phase of the wave function of a system. The Lagrangian is then defined as the rate of change of the phase and given as

\begin{equation}
	\mathcal{S} = \int_{t_i \mathcal{P}}^{t_f} L \textrm{d}t,
\end{equation}

where we integrate over all intermediate states taken by the system. The principle of least action then states that the actual path taken by the system is one that minimizes the action, or more precicely one where the action is stationary. Hamilton's principle then states that the variation of the line integral going from $t_i$ to $t_f$ is zero\cite{goldstein1959};

\begin{equation}
	\delta \mathcal{S} = \delta \int_{t_i}^{t_f} L \textrm{d}t = 0.
\end{equation}

Using this principle, and knowing the Lagrangian of the system, one can derive the equations of motion of the system.

\subsubsection{The path integral formalism}
In 1948 Richard Feynman presented his new formulation of non relativistic quantum mechanics using what is know as the path integral formalism\cite{feynman1948sta}. The path integral takes in to account all possible paths that a particle, or any other quantum mechanical system, going from a state $\ket{i}$ to a state $\ket{f}$ can take. This can be changes in position, momentum, energy or other intrinsic variables and is represented as an intermediate state of the system. Generally, and this is actually a central point in quantum mechanics, we are not interested in the path taken, but only in the fact that at some time $t_{i}$ the particle was in a state $\ket{i}$ and at some later time $t_{f}$ the particle is in a state $\ket{f}$. In order to arrive at this quantum probability-amplitude we must sum, or integrate, over the infinitely many paths, that the system can take to go from initial to final state. Doing this the amplitude is given as \cite{richter_path_integrals}

\begin{equation}
	\bra{f} U(t_f,t_i) \ket{i} = \mathcal{N} \int \mathcal{D}x e^{i\mathcal{S}},
\end{equation}

where $\mathcal{N}$ is a constant of the system, $U$ is the time translation operator, $\mathcal{D}$ is an abbreviation given by the relation 

\begin{equation}
	\int \mathcal{D} x \equiv \lim_{N \to \infty} \int dx_1 \dots dx_N
\end{equation}

and the action $\mathcal{S}$ is related to the Lagrangian by the action principle, given in the 1-D case by

\begin{equation}
	\mathcal{S} = \int_{t_i}^{t_f} L(x,\dot{x},t)dt,
\end{equation}
where $x, \dot{x} \equiv \frac{dx}{dt}$ are generalized coordinates.

\subsubsection{Quantum field theory}
Going from classical to quantized fields.

\subsubsection{Gauge transformations and gauge groups}
Symmetries and Noethe's Theorem.

\subsection{The Standard Model}

The particles are divided into two main groups; fermions and bosons. Fermions are defined as having half-integral spin and are described by Fermi-Dirac statistics, these include the leptons and quarks. Fermions constitutes all known matter and as such they are sometimes described as matter-particles. Bosons, on the other hand, have zero or integral spin and are described by Bose-Einstein statistics. Some of these, namely the gauge bosons, are responsible for the weak, strong and electromagnetic interactions. Therefore the gauge bosons are often called the force-carriers of their respective interactions. It may be appropriate to note that the formalism of quantum mechanics makes no clear distinction between the concepts of matter-particles and force -particles.

% Billede med partikel-generationer fra Wikipedia.
\includefigure{fig:particle_generations}{0.3}{./images/particle_generations.jpg}{The fundamental particles in the Standard Model. Fermions and bosons, listed in their respective generations. (Source: Wikimedia Commons).}

\subsubsection{The gauge groups of the SM}
The gauge boson mediating the electromagnetic force is the photon ($\gamma$). The theory is derived from the U(1) gauge group, which is just the group of phase rotations.

\begin{equation}
    \psi \rightarrow e^{i\alpha} \psi
\end{equation}

The weak interaction is derived from the SU(2) group of unitary matrices with determinant 1. The gauge boson associated with this group are the $W^+$, $W^-$ and the $Z^0$ bosons.

Combining these two theories we arrived at what is know as Quantum Electrodynamics (QED). The theory which is described by the gauge group (U(1) $\times$ SU(2)).

The gauge bosons mediating the strong interaction are characterized by the SU(3) gauge group. The generators of which are the 8 Gell-Mann matrices giving rise to 8 gauge bosons of the strong interaction know as gluons, each having a property called color. Because of this color feature the gauge theory of the strong interaction is called Quantum Chromodynamics (QCD).

Combining QED and QCD into U(1) $\times$ SU(2) $\times$ SU(3) we arrive at what is know as the Standard Model (SM) of particle physics.

But one important feature is still missing.

%%%%%%%%%%%%%%%%%%%%%%%%%%%%%%%%%%%%%%%%%%%%%%%%%%%%%%%%%%%%%%%%%%%%%%%%
%%%%%%%%%%%%%%%%% HER FØLGER TEST AF FEYNMAN-DIAGRAMMER %%%%%%%%%%%%%%%%
%%%%%%%%%%%%%%%%%%%%%%%%%%%%%%%%%%%%%%%%%%%%%%%%%%%%%%%%%%%%%%%%%%%%%%%%

\subsubsection{Test Feynman Diagrammer}

\begin{figure}[!htb]
\begin{center}
\begin{tabular}{cccccccccccccccc}    %Adds several centered Columns
	
	
	\begin{fmffile}{one} 	%one.mf will be created for this feynman diagram  
	  \fmfframe(1,7)(1,7){ 	%Sets dimension of Diagram
	   \begin{fmfgraph*}(110,62) %Sets size of Diagram
	    \fmfleft{i1,i2}	%Sets there to be 2 sources 
	    \fmfright{o1,o2}    %Sets there to be 2  outputs
	    \fmflabel{$e^-$}{i1} %Labels one of the left sources
	    \fmflabel{$e^+$}{i2} %Labels one of the left sources
	    \fmflabel{${\ensuremath{\erlpm}}$}{o1} %Labels one of the right outputs
	    \fmflabel{${\ensuremath{\erlpm}}$}{o2} %Labels one of the right outputs
	    \fmf{fermion}{i1,v1,i2} %Connects the sources with a vertex.
	    \fmf{fermion}{o1,v2,o2} %Connects the outputs with a vertex.
	    \fmf{photon,label=$\gamma/Z^0$}{v1,v2} %Labels the conneting line.
	   \end{fmfgraph*}
	  }
	\end{fmffile}
	&&&&
	
	\begin{fmffile}{two}
	  \fmfframe(1,7)(1,7){ 
	   \begin{fmfgraph*}(110,62)
	    \fmfleft{i1,i2}
	    \fmfright{o1,o2}
	    \fmflabel{$e^-$}{i1}
	    \fmflabel{$e^+$}{i2}
	    \fmflabel{${\ensuremath{\erlpm}}$}{o1}
	    \fmflabel{${\ensuremath{\erlpm}}$}{o2}
	    \fmf{fermion}{i1,v1,o1}
	    \fmf{fermion}{i2,v2,o2}
	    \fmf{photon,label=$\chionez$}{v1,v2}
	   \end{fmfgraph*}
	  }
	\end{fmffile}
	\end{tabular}
	\caption{S-Channel left, T-Channel right}\label{fey1}
	\end{center}
	\end{figure}

