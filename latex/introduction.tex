\section{Introduction In English}
In describing fundamental physical phenomena, physicists make use of the concept of a particle representing a certain small-scale state (of the order $10^{-15}$ m) of the universe. Particles can interact with each other annihilating or creating new particles (or states). To account for these phenomena we use make use of the notion of interaction between the particles. In particle physics the description of an interaction accounts for the particles involved, their intrinsic characteristics, conversed quantities, information about the before/ after state, and so on.

With these two concepts at hand we can go on and build mathematical frameworks based on experimental results and/or purely mathematical ideas, with the intend to extend our knowledge of the Universe and increase the predictive power and accuracy of the theories involved. Using this method physicists have been able to create an extensive theoretical framework of amazing predictive accuracy. This framework, known as the Standard Model (SM) of particle physics, is written in the mathematical language of Quantum Field Theory (QFT). In the context of QFT particles are thought of as excitations in underlying fields propagating through spacetime.

Another accurate framework was developed in the beginning of the last century by german physicist Albert Einstein. This framework, describing gravity on large scales, is know as the General Theory of Relativity (GR). Together with the SM these two theories represent our best current knowledge of the physical Universe at the very large and very small scales.\footnote{Henceforth when we write knowledge the reader may assume this is equivalent to the information available from being able to predict the time evolution of physical systems or states.}

These two frameworks, although very useful in their own domains, are mutually incompatible. In GR spacetime is thought of as a smooth and continuous manifold, whilst spacetime in QFT is a turmoil of fluctuating fields. From a macroscopic perspective this might not seem like a problem, after all the fluctuations are tiny compared to every-day energy scales. The problem arises when we try to calculate the effects of gravity at the quantum scale; we end up with horribly divergent results and thereby infinities.

To try and solve these problems, physicists have developed several new theoretical candidates that attempt to avoid infinities, when dealing with quantum gravity (QG), by defining a so called minimal length scale. By doing this they hope to minimize the quantum fluctuation's effect on the path integral formalism.
% ER DET RIGTIGT FORSTÅET!?
% Kan man ikke sige "quantum fluctuation's effects" på en finere måde?

In the following report we will concentrate on one of these candidates, namely the extension of the SM by Non-Commutative Geometry (NCG). We will discuss how this theory attempts to avoid infinities in QG and explore some of the many consequences that NCG has. More specifically we will see that the theory predicts new vertices between particles that are illegal in the normal context of the SM. These vertices arise from interactions between the neutral gauge boson Z$^0$, responsible for weak-force decays, and the charged gauge bosons the gluons ($g$), responsible for strong-force decays.

We also analyse the particular conditions necessary for the interactions to be observed in experiments such as those happening at the Large Electron-Positron Collider (LEP) and the Large Hadron Collider (LHC). Furthermore we explore the different constraints that current experimental data puts on the detailed structure of this particular geometrical modification of the SM.
% Kan man ikke godt sige at det er en geometrisk modifikation?

\newpage
\section{Introduction In Danish}