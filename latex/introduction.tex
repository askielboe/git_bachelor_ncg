\section{Introduction}
The physics describing the phenomena that surrounds us in our daily life is classical mechanics. When things get very small, compared with the size of atoms, or move really fast, compared with the speed of light, the predictive accuracy of the Newton's laws break down. Therefore we need other theories to describe the observed phenomena in these regions.

In describing fundamental physical phenomena, physicists make use of the concept of a particle, representing a certain small-scale state of the universe. This scale is of the order $10^{-15}$ m and is much too small to be probed by ordinary laboratory equipment like microscopes. Particles can influence with each other by scattering or annihilating, thereby creating new states. To account for these phenomena we make use of the notion of an interaction between the particles. A complete description of an interaction in particle physics should account for all the particles involved, their intrinsic characteristics, conserved quantities, information about the state before and after the interaction, and so on.

With these two concepts at hand, we can go on and build mathematical models based on experimental results and/or purely mathematical ideas. This allows us to extend our knowledge of the Universe and increase our predictive power and accuracy of the theories involved. Using this method, physicists have been able to create an extensive theoretical framework, describing the interactions of elementary particles to an amazing level of accuracy. The framework, known as the Standard Model (SM) of particle physics, is written in the mathematical language of Quantum Field Theory (QFT). In the context of QFT particles are thought of as excitations in underlying fields propagating through spacetime.

Another accurate framework was developed in the beginning of the last century by German physicist Albert Einstein. This framework, describing gravity on large scales, is know as the General Theory of Relativity (GR). Together with the SM these two theories currently represent some of our best knowledge of the physical Universe at the greatest and smallest scales.

These two frameworks, although very useful in their own domains, are mutually incompatible. In GR, spacetime is thought of as a smooth and continuous manifold, whilst QFT, spacetime is a turmoil of fluctuating fields. [HEY HJÆLP JØRGEN!?!?!?!?!?!?!?!?!?!?!?!!?!?!??!!!!!!!!!!!!!????????????????????????????????????????????????????????????] From a macroscopic perspective this might not seem like a problem; after all the fluctuations are tiny compared to every-day energy scales. The problem arises when we try to calculate the effects of gravity at the quantum scale; we end up with horribly divergent results and our, otherwise consistently accurate, theory breaks down.

To try and solve some of these problems, physicists have developed several new theoretical candidates that attempt to avoid infinities, when dealing with quantum gravity (QG), by defining a so called minimal length scale.
% By doing this they hope to minimize the quantum fluctuation's effect on the path integral formalism.
% ER DET RIGTIGT FORSTÅET!?
% Kan man ikke sige "quantum fluctuation's effects" på en finere måde?

In the following report we will concentrate on one of these candidates, namely the extension of the SM by Non-Commutative Geometry (NCG). We will discuss how this theory attempts to avoid infinities in QG and explore some of the many consequences that NCG has. Specifically we will study new interactions, that the theory predicts the existence of, between particles. Interactions that are not seen within the ordinary framework of the SM. These vertices arise from interactions between the neutral gauge boson Z$^0$, responsible for weak-force decays, and color charged gauge bosons, called gluons ($g$), responsible for strong-force interactions.

We also analyze the scale of the model necessary for the interactions to be observed in experiments such as those happening at the Large Electron-Positron Collider (LEP) and the Large Hadron Collider (LHC). Furthermore, we explore the different constraints that current experimental data puts on the scale of this particular geometrical modification of the SM.
% Kan man ikke godt sige at det er en geometrisk modifikation?
