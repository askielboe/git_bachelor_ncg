\section{Introduction}
The physics describing the phenomena that surrounds us in our daily life, is classical mechanics. When things get very small, compared with the size of atoms, or move really fast, compared with the speed of light, the predictive accuracy of the Newtonian laws breaks down. Therefore we need other theories to describe the observed phenomena in these regions.

In describing fundamental physical phenomena, physicists make use of the concept of a particle representing a certain small-scale state of the universe. This scale is of the order $10^{-15}$ m and much too small to be probed by ordinary laboratory equipment like microscopes. Particles can interact with each other, by annihilating or creating new states. To account for these phenomena we make use of the notion of an interaction between the particles. In particle physics the description of an interaction accounts for the particles involved, their intrinsic characteristics, conversed quantities, information about the state before and after the interaction and so on.

With these two concepts at hand we can go on and build mathematical frameworks based on experimental results and/or purely mathematical ideas, with the intend to extend our knowledge of the Universe and increase the predictive power and accuracy of the theories involved. Using this method physicists have been able to create an extensive theoretical framework of amazing predictive accuracy. This framework, known as the Standard Model (SM) of particle physics, is written in the mathematical language of Quantum Field Theory (QFT). In the context of QFT particles are thought of as excitations in underlying fields propagating through spacetime.

Another accurate framework was developed in the beginning of the last century by german physicist Albert Einstein. This framework, describing gravity on large scales, is know as the General Theory of Relativity (GR). Together with the SM these two theories represent our best current knowledge of the physical Universe at the very large and very smallest scales.

These two frameworks, although very useful in their own domains, are mutually incompatible. In GR spacetime is thought of as a smooth and continuous manifold, whilst in QFT spacetime is a turmoil of fluctuating fields. From a macroscopic perspective this might not seem like a problem, after all the fluctuations are tiny compared to every-day energy scales. The problem arises when we try to calculate the effects of gravity at the quantum scale; we end up with horribly divergent results and our, otherwise consistently accurate, theory breaks down.

To try and solve these problems, physicists have developed several new theoretical candidates that attempt to avoid infinities, when dealing with quantum gravity (QG), by defining a so called minimal length scale.
% By doing this they hope to minimize the quantum fluctuation's effect on the path integral formalism.
% ER DET RIGTIGT FORSTÅET!?
% Kan man ikke sige "quantum fluctuation's effects" på en finere måde?

In the following report we will concentrate on one of these candidates, namely the extension of the SM by Non-Commutative Geometry (NCG). We will discuss how this theory attempts to avoid infinities in QG and explore some of the many consequences that NCG has. More specifically we will see that the theory predicts new vertices between particles that are not seen within the ordinary framework of the SM. These vertices arise from interactions between the neutral gauge boson Z$^0$, responsible for weak-force decays, and color charged gauge bosons, the gluons ($g$), responsible for strong-force interactions.

We will also analyze the particular conditions necessary for the interactions to be observed in experiments such as those happening at the Large Electron-Positron Collider (LEP) and the Large Hadron Collider (LHC). Furthermore we explore the different constraints that current experimental data puts on the detailed structure of this particular geometrical modification of the SM.
% Kan man ikke godt sige at det er en geometrisk modifikation?

\newpage
\section{Introduktion}
Den fysik der omgiver os i dagligdagen kan beskrives med klassisk mekanik. Når fænomener bliver meget små, sammenlignet med størrelsen af et atom, eller bevæger sig meget hurtigt, sammenlignet med lysets hastighed, så holder Newtons love op med at virke. Derfor behøver vi andre teorier til at beskrive de observationer vi gør i disse regioner.

Når fysikere skal beskrive de fundamentale fysiske fænomener bruger de to koncepter. Det første er begrebet partikel, dette beskriver en given tilstand på meget lille skala, $10^{-15}$m. Partikler kan interagere så der dannes nye tilstande ved at partikler skabes eller annihileres. Fysikere kalder denne interaktion for en vekselvirkning. I partikelfysik beskriver man en vekselvirkning mellem partikler ved partiklernes begyndelses- og sluttilstande og de bevarede størrelser.

Med disse to koncepter, partikler og vekselvirkninger, kan vi skabe den matematiske ramme der kan hjælpe eksperimentalfysikerne med at beskrive de resultater der observeres, og teoretikerne med at formulere de rene matematiske idéer der tilsammen skal udvide vores horisont og viden om Universet. Med denne metode har fysikere skabt en omfattende teoretisk ramme, der indtil nu har haft den største eksperimentelle forudsigelseskraft. Denne ramme kaldes Standard Modellen (SM) og er skrevet i et matematisk sprog båret af kvantefeltteorien (KFT). I KFT beskrives partikler som eksitationer af et underliggende felt der udbreder sig i tiden.

En anden teoretisk ramme til beskrivelsen af de fundamentale fysiske fænomener er Generel Reletavitetsteori (GR). Denne blev udviklet af Albert Einstein i begyndelsen af det sidste århunderede. GR beskriver tyngdekraften på stor skala. SM og GR repræsenterer det nutidige bedste bud på de teorier der beskriver vores viden om det fysiske univers på både stor og lille skala.

Disse to teorier er, på trods af deres store anvendelighed i deres domæner, desværre ikke forenelige. I GR er rumtiden en kontinuerlig og glat manifold, mens den i KFT skal forstås som et tumult af fluktuerende felter. Fra et makroskopisk synspunkt virker dette måske ikke som et problem, da man kan hævde at disse fluktuationer er meget små i sammenligning med hverdagsfænomener. Problemet kommer først når vi prøver at regne på de effekter tyngdekraften har på kvanteskalaen, her ender vi med at have utroligt divergerende løsninger og derfor uendeligheder.

For at løse disse problemer har fysiskere udviklet en række teoretiske kandidater der skal hindre disse uendeligheder i at dukke op når der regnes på kvantegravitaioner. Dette gøres ved at indføre en minimum længdeskala og dermed hindre kvantefluktuationernes effekt på "path integral" formalismen.

I den følgende opgave vil vi koncentrere os om én af disse kandidater, udvidelsen af SM med Ikkekommutativ Goemetri (NCG). Vi vil diskutere hvordan denne teori hindrer uendeligheder i kvantegravitation og hvilke konsekvenser det har at indføre NCG. Mere specifikt vil vi se hvordan teorien forudsiger nye koblinger mellem partikler, der ifølge SM er forbudte. Helt specifikt vil vi se på koblingen mellem den neutrale gauge boson, Z$^{0}$, den der er ansvarlig for den svage kernekraft, og de ladede gauge bosoner, gluoner ($g$), dem der er ansvarlige for den stærke kernekraft.

Vi vil også kigge på muligheder for at dette henfald kan være blevet observeret ved Large Electron-Positron Collider (LEP) og blive observeret ved Large Hadron Collider (LHC). Ydermere vil vi kigge på de restriktioner der kommer fra nuværende eksperimentelle data på den detaljerede struktur af denne geometriske modifikation af SM som NCG medfører.
