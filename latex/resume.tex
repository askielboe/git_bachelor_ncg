\section{Resumé på dansk}
Forhåbentlig vil LHC starte denne sommer og data vil begynde at tegne et billede af de energiområder som aldrig før har været undersøgt. To protoner vil kollidere i 4 forskellige detektorer med en energi på 7 TeV hver. Dette vil i bedste fald føre til en masse viden om den fysik der hersker på disse niveauer. Et af de store spørgsmål som søger svar i denne forbindelse er foreningen af Generel Relativitetsteori (GR) og Kvante Felt Teori (QFT). I dette bachelor projekt har vi arbejdet med Ikkekommutativ Geometri (NCG). NCG er én af kandidaterne til den matematiske fremstilling af en teori der forener GR og QFT ved at definerer en minimum længdeskala. Vi har lavet et litterært studie af de concepter der er nødvendige for at forstå idéen bag NCG. Dette indebærer viden om QFT, GR og Standard Modellen (SM). En af konsekvenserne ved at udvide Standard Modellen (SM) med NCG er et nyt vertex hvor gluoner kobler til Z-bosonen. Dette nye bidrag til det totale tværsnit for Z har vi undersøgt udfra den antagelse at bidraget kan justeres ved en parameter $\Lambda$. Vi har fundet to bud på en nedre grænse for $\Lambda$, det første bud er fundet med data for Z-tværsnittet ved LEP og det andet bud er fundet ved at simulere muonproduktionen ved LHC med programmet CompHEP. 

Vi antager at bidraget til Z-bredden kommende fra NCG må være mindre end usikkerheden på det totale Z-tværsnit fundet ved LEP. Dette forudsiger at $\Lambda$ skal være større end 117 GeV, se figur \ref{fig:kplot}. 

Ved at simulere den totale muonproduktion ved LHC, og igen antage at bidraget til muonproduktionen kommende fra NCG må være mindre end usikkerheden på det totale antal, kan vi plotte for forskellige værdier af $\Lambda$ og derigennem finde at $\Lambda$ skal være større end 900 GeV, se figur \ref{fig:lambdaplot}. 
