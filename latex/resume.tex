\section{Resumé på dansk}
Forhåbentlig vil LHC starte denne sommer og data vil begynde at tegne et billede af de energiområder som aldrig før har været undersøgt. To protoner vil kollidere i 4 forskellige detektorer med en energi på 7 TeV hver. Dette vil i bedste fald føre til en masse viden om den fysik der hersker på disse niveauer. Et af de store spørgsmål som søger svar i denne forbindelse er foreningen af Generel Relativitetsteori (GR) og Kvante Felt Teori (QFT). I dette bachelor projekt har vi arbejdet med Ikkekommutativ Geometri (NCG). NCG er én af kandidaterne til den matematiske fremstilling af en teori der forener GR og QFT ved at definerer en minimum længdeskala. Vi har lavet et litterært studie af de concepter der er nødvendige for at forstå idéen bag NCG. Dette indebærer viden om QFT, GR og Standard Modellen (SM). En af konsekvenserne ved at udvide Standard Modellen (SM) med NCG er et nyt vertex hvor gluoner kobler til Z-bosonen. Dette nye bidrag til det totale tværsnit for Z har vi undersøgt udfra den antagelse at bidraget kan justeres ved en parameter $\Lambda$. Ved hjælp af data fra LEP har vi undersøgt ved hvilken værdi af $\Lambda$ bidraget fra NCG er så småt at det ligger inden for usikkerheden for det totale Z tnærsnit. Hvis bidraget er større ville vi jo have haft set det. Vi fandt at $\Lambda$ skal være større end 117 GeV. Med denne viden har vi lavet plots over den forudsigelige muonproduktion, fundet ved LHC, med forskellige værdier for $\Lambda$, henholdsvis 500, 1000, 1500 og 2000 GeV. Vi ser at det beregnede antal med bidraget fra det nye vertex er alt for stort....