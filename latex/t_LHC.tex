\subsection{Experimental high energy physics at LHC}
The Large Hadron Collider (LHC) is the worlds highest-energy and also largest particle accelerator. It is designed for hadron-hadron collisions with each beam having an energy of 7 TeV. LHC has four intersection points and the detectors at these points are respectively ATLAS, ALICE, CMS and LHC-b. ATLAS and CMS are the ones which will look for the Higgs boson and new physics like extra dimensions, the nature of dark matter, Super Symmetry and maybe NCG. Each bunch has 1,15·10$^{11}$ protons. 

\subsubsection{Luminosity}
The luminosity is the number of particles passing per unit time, per unit area. It is given by:

\begin{equation}
L=n\frac{N_{1}N_{2}}{A}f
\end{equation}

where $n$ is the number of colliding bunches, $N_{1}$ and $N_{2}$ are the number of particles in each bunch, $A$ is the cross sectional area of the beam and $f$ is the frequency of the beam. Therefore L quantifies the intensity of the beams and the density of the particles in each beam and tells us how many number of events the collider can produce in a given time. If A gets smaller the luminosity will be greater and therefore the beam is squeezed when it approach a detector. LCH is initially set to start-up with a luminosity of 10 fb$^{-1}$ in a year.

\subsubsection{Cross section}
The cross section is a measure of the probability of a given process occurring in the accelerator. So, larger cross-sections mean that a process is more likely to occur. You can think of the situation where you wanna throw two things against each other, the probability of how good the they will hit, depends on their cross-sectional area. Therefore Cross-sections are measured in barns, wich equals 10$^{-28}$ m$^{2}$



and is given by:

\begin{equation}
\sigma_{tot}=\sum_{i=1}^n \sigma_{i} \qquad \sigma_{i}=
\end{equation}

\subsubsection{Parton Distribution Functions, PDF}
During the collision between two protons, the proton will disassociate into a virtual state of u and d quarks, gluons and a "sea" of quark and anti-quark called partons. With very high collision energy, we can ignore the quantum effect and the partons can be considered to be free. Therefore, the partons scatter incoherently and the proton cross section is simply the sum of the individual parton cross sections. This is a classical probability distribution function, given by $f_{i}(x)$, where $i$ refers to the different types of the partons and $x$ is the fraction of the proton momentum carried by the parton\cite{green2005hpp}. $f_{i}(x)$ is therefore the probability distribution to find a parton with momentum fraction $x$. PDF's are made from global fits to experimental data.

