\subsection{Experimental high energy physics at LHC}
The Large Hadron Collider (LHC) 


cross section




\subsubsection{Parton Distribution Funktion, PDF}
During the collision between two protons, the proton will dissasociate into a virtual state of u and d quarks, gluons and a "sea" of quark and antiquark called partons. With very high collision energy, we can ignore the quantum effect and the partons can be considered to be free. Therefore, the partons scatter incoherently and the proton cross section is simply the sum of the individual parton cross sections. This is a classical probability distribution function, given by $f_{i}(x)$, where $i$ refers to the different types of the partons and $x$ is the fraction of the proton momentum carried by the parton\cite{green2005hpp}. $f_{i}(x)$ is therefore the probability distribution to find a parton with momentum fraction $x$.


unfortunately





The Large Hadron Collider (LHC) is the world's largest and highest-energy particle accelerator, intended to collide opposing particle beams, of either protons at an energy of 7 TeV per particle, or lead nuclei at an energy of 574 TeV per nucleus.

The Large Hadron Collider was built by the European Organization for Nuclear Research (CERN) with the intention of testing various predictions of high-energy physics, including the existence of the hypothesized Higgs boson[1] and of the large family of new particles predicted by supersymmetry.[2] It lies in a tunnel 27 kilometres (17 mi) in circumference, as much as 175 metres (570 ft) beneath the Franco-Swiss border near Geneva, Switzerland. It is funded by and built in collaboration with over 10,000 scientists and engineers from over 100 countries as well as hundreds of universities and laboratories.[3]

On 10 September 2008, the proton beams were successfully circulated in the main ring of the LHC for the first time.[4] On 19 September 2008, the operations were halted due to a serious fault between two superconducting bending magnets.[5] Due to the time required to repair the resulting damage and to add additional safety features, the LHC is scheduled to be operational again no sooner than September 2009.[6]



In addition to the Higgs boson, new particles predicted by possible extensions of the Standard Model might be produced at the LHC

[DEFINE THE CONCEPT OF CROSS SECTION]

