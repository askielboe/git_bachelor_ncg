\subsection{Experimental high energy physics at LHC}
The Large Hadron Collider (LHC) is the worlds highest-energy and also largest particle accelerator. It is designed for hadron-hadron collisions with each beam having an energy of 7 TeV. LHC has four intersection points and the detectors at these points are respectively ATLAS, ALICE, CMS and LHC-b. ATLAS and CMS are the ones which will look for the Higgs boson and new physics like extra dimensions, the nature of dark matter, Super Symmetry and maybe NCG. 

\subsubsection{Luminosity}
The luminosity is the number of particles passing per unit time, per unit area. It is given by
\begin{equation}
L=n\frac{N_{1}N_{2}}{A}f,
\end{equation}
where $n$ is the number of colliding bunches, $N_{1}$ and $N_{2}$ are the number of particles in each bunch, $A$ is the cross sectional area of the beam and $f$ is the frequency of the bunches passing through the entire ring \cite{martin1998pp}. L then quantifies the intensity of the beams and the particle density in each beam and tells us how many number of events the collider can produce in a given time. If A becomes smaller the luminosity becomes larger. The LHC is initially set at a luminosity of 10 fb$^{-1}$ per year[CITATION].

\subsubsection{Cross section}
The cross section is a measure of the probability of a given process occurring in the accelerator. Larger cross-sections means that a process is more likely to occur. You can think of the situation where you want to throw two things against each other, the probability for them to hit each other, depends on their cross-sectional area. Therefore cross-sections are measured in area, in particle physics, using the unit of barn. One barn is equal to 10$^{-28}$ m$^{2}$.
The cross section depends on the nature and structure of the two colliding beams, as you can see in Fermi's golden rule \eqref{eq:goldenrule}. The amplitude for the process $\mathcal{M}$ contains all the dynamical information. If we recall the Feynman rule for a propagator: $\frac{i}{p^{2}-(mc)^{2}}$ we notice that when the momenta of the incoming particles approach the rest mass of the propagator the cross section blows up. This is what we call a 'resonance', a special energy where particles involved are more likely to interact. This is why the cross section is such a useful parameter, because it contains a lot knowledge about the physics going on. Each of the possible interactions have a given cross section ($\hat \sigma(\hat s)$), which is a function of $\hat s$, the energy availably in the interaction. The total cross section is simply the sum of all the cross sections of the interactions involved\cite{griffiths1987iep}.

\subsubsection{Parton Luminosity Functions, $\mathcal{L}^{pp}$}
During the collision between two protons ($p$), the proton will disassociate into a virtual state of u and d quarks, gluons and a "sea" of quark and anti-quark called partons. With very high collision energy, we can ignore the quantum effect and the partons can be considered to be free. Therefore, the partons scatter incoherently and the proton cross section is simply the sum of the individual parton cross sections. This is a classical probability distribution function, given by $f_{i}(x)$, where $i$ refers to the different types of the partons and $x$ is the fraction of the proton momentum carried by the parton\cite{green2005hpp}. $f_{i}(x)$ is therefore the probability distribution to find a parton with momentum fraction $x$. The parton luminosity functions is the probability of taking a parton from each beam with fraction respectively $x_{1}$ and $x_{2}$:

\begin{equation}
\mathcal{L}^{pp}(\hat s)=\int\limits_{0}^{1} f_{a}(x_{1}) f_{b}(x_{2})\, dx_{1}dx_{2}\cdot \delta(\hat s - x_{1}x_{2}s)
\end{equation}

the delta-function ensures that $\hat s = x_{1}x_{2}s$. If we put the cross section and the parton luminosity function together we get the total cross section:

\begin{equation}
\sigma_{total}=\int \sigma(\hat s) \, d \hat s \cdot \int\limits_{0}^{1} f_{a}(x_{1}) f_{b}(x_{2})\, dx_{1}dx_{2} \cdot \delta(\hat s - x_{1}x_{2}s)
\end{equation}

in our further work we approximate this by the expression:

\begin{equation}
\sigma_{total}= \hat \sigma(\hat s) \otimes  \mathcal{L}_{pp}(\hat s)
\end{equation}

where $\otimes$ means that we multiply the two distribution, bin by bin. In the limit with infinite number of bins the two expressions are equal. \\

As you might sense the luminosity, cross section and parton luminosity is the key tools when operating with high energy physics. The number of events is given as $N=L\cdot\sigma_{total}$ and therefore we get more events with higher L and $\sigma_{total}$. This is why the beam is squeezed when it approach a detector and the accelerator is tuned to the energies of the resonance of interest.

