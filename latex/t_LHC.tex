\subsection{Experimental high energy physics at LHC}
The Large Hadron Collider (LHC) is the worlds highest-energy and also largest particle accelerator. It is designed for proton-proton collisions with each beam having an energy of 7 TeV. LHC has four intersection points and the detectors at these points are respectively ATLAS, ALICE, CMS and LHC-b. ATLAS and CMS are the ones which will look for the Higgs boson and new physics like extra dimensions, the nature of dark matter, Super Symmetry (SUSY) and potentially NCG.

\subsubsection{Luminosity}
The instant luminosity is the number of particles passing per unit time, per unit area. It is given by
\begin{equation}
	L=n\frac{N_{1}N_{2}}{A}f \quad [\mbox{cm}^{-2}\mbox{s}^{-1}],
\end{equation}
where $n$ is the number of colliding bunches, $N_{1}$ and $N_{2}$ are the number of particles in each bunch, $A$ is the cross sectional area of the beam and $f$ is the frequency of the bunches passing through the entire ring \cite{martin1998pp}. L then quantifies the intensity of the beams and the particle density in each beam and tells us how many number of events the collider can produce in a given time. If A becomes smaller the luminosity becomes larger. The LHC is initially set at luminosity of $10^33$ cm$^{-2}$s$^{-1}$ corresponding to an integrated luminosity of $10$ fb$^{-1}$ per year.

\subsubsection{Cross section}
The cross section is a measure of the probability of a given process occurring in the accelerator. Larger cross-sections means that a process is more likely to occur. You can think of the situation where you want to throw two things against each other, the probability for them to hit each other, depends on their cross-sectional area. Therefore cross-sections are measured in area, in particle physics, using the unit of barn. One barn is equal to 10$^{-28}$ m$^{2}$.
The cross section depends on the nature and structure of the two colliding beams, as you can see in Fermi's golden rule \eqref{eq:goldenrule}. The amplitude for the process $\mathcal{M}$ contains all the dynamical information. If we recall the Feynman rule for a propagator: $\frac{i}{p^{2}-(mc)^{2}}$ we notice that when the momenta of the incoming particles approach the rest mass of the propagator the cross section blows up. This is what we call a 'resonance', a special energy where particles involved are more likely to interact. This is why the cross section is such a useful parameter, because it contains a lot knowledge about the physics going on. Each of the possible interactions have a given cross section ($\hat \sigma(\hat s)$), which is a function of $\hat s$, the energy availably in the interaction. The total cross section is simply the sum of all the cross sections of the interactions involved\cite{griffiths1987iep}.

\subsubsection{Parton Luminosity Functions, $\mathcal{L}^{pp}$}
During the collision between two protons, the proton will disassociate into a virtual state of u and d quarks, gluons and a "sea" of quarks and anti-quarks called partons. Because of the principle of asymptotic freedom in QCD, we can, at sufficiently high collision energy, ignore the effects of confinement, and consider the partons to be free non-interacting particles. Therefore, the partons scatter incoherently and the proton cross section is simply the sum of the individual parton cross sections. This is a classical probability distribution function, given by $f_{i}(x)$, where $i$ refers to the different types of the partons and $x$ is the fraction of the proton's momentum carried by the parton \cite{green2005hpp}. $f_{i}(x)$ is then the probability distribution related to finding a parton with momentum that is a fraction $x$ of the momentum of the proton. The parton luminosity functions describe the probability of taking a parton from each beam with fractions $x_{1}$ and $x_{2}$ respectively
\begin{equation}
\mathcal{L}^{pp}(\hat s)=\int\limits_{0}^{1} f_{a}(x_{1}) f_{b}(x_{2})\, dx_{1}dx_{2}\cdot \delta(\hat s - x_{1}x_{2}s).
\end{equation}
The delta-function ensures that $\hat s = x_{1}x_{2}s$. If we put the cross section and the parton luminosity function together we get the total cross section given by

\begin{equation} \label{eq:lumicrosssection1}
\sigma_{total}=\int \sigma(\hat s) \, d \hat s \cdot \int\limits_{0}^{1} f_{a}(x_{1}) f_{b}(x_{2})\, dx_{1}dx_{2} \cdot \delta(\hat s - x_{1}x_{2}s).
\end{equation}

In our further work we approximate this by

\begin{equation} \label{eq:lumicrosssection2}
\sigma_{total}= \hat \sigma(\hat s) \otimes  \mathcal{L}_{pp}(\hat s),
\end{equation}

where $\otimes$ means that we multiply the two distributions, bin by bin. In the limit where the number of bins goes to infinity the two expressions, \eqref{eq:lumicrosssection1} and \eqref{eq:lumicrosssection2} are equal. The luminosity, cross section and parton luminosity are key tools when studying high energy physics from proton collisions. The number of events is given as $N=L\cdot\sigma_{total}$.