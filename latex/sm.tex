\section{The Standard Model}
The particles and interactions of the SM are described in the language of Quantum Mechanics (QM) and Quantum Field Theory (QFT), which have a natural way of unifying the mathematical concepts in a concise way.

The particles are divided into two main groups; fermions and bosons. Where fermions are defined as having half-integral spin and are described by Fermi-Dirac statistics, these include the leptons and quarks. Fermions constitutes all known matter and as such they are sometimes described as matter particles. Bosons on the other hand have zero or integral spin and are described by Bose-Einstein statistics. Some of these, namely the gauge bosons, are responsible for the weak, strong and electromagnetic interactions. Therefore the gauge bosons are often called the force-carriers of their respective interactions. It may be appropriate to note that the formalism of quantum mechanics makes no clear distinction between the concepts of matter and force -particles.

% Billede med partikel-generationer fra Wikipedia.
\includefigure{fig:particle_generations}{0.3}{./images/particle_generations.jpg}{The fundamental particles in the Standard Model. Fermions and bosons, listed in their respective generations. (Source: Wikimedia Commons).} 

\subsection{Gauge transformations}

\subsection{Gauge groups}
The gauge boson mediating the electromagnetic force is the photon ($\gamma$). The theory is derived from the U(1) gauge group, which is just the group of phase rotations.

\begin{equation}
    \psi \rightarrow e^{i\alpha} \psi
\end{equation}

The weak interaction is derived from the SU(2) group of unitary matrices with determinant 1. The gauge boson associated with this group are the $W^+$, $W^-$ and the $Z^0$ bosons.

Combining these two theories Weinberg and Salam [DET VAR DA IKKE Weinberg \& Salam!?] arrived at what is called Quantum Electrodynamics (QED) which is described by the gauge group U(1) $\times$ SU(2).

The gauge bosons mediating the strong interaction are characterized by the SU(3) gauge group. The generators of which are the 8 Gell-Mann matrices giving rise to 8 gauge bosons of the strong interaction know as gluons, each having a property called color. Because of this color feature the gauge theory of the strong interaction is called Quantum Chromodynamics (QCD).

Combining QED and QCD into U(1) $\times$ SU(2) $\times$ SU(3) we arrive at what is know as the Standard Model (SM) of particle physics.

But one important feature is still missing.

\subsection{Quantum Gravity}
General Relativity (GR) is the theory describing


So far there have been no successful unification of the SM and General Relativity (GR), which is the theory describing all know macroscopic