\typeout{- START ----------------------------------- Preamble.tex --}
\documentclass[11pt,a4paper,titlepage]{article}
%\usepackage[danish]{babel}
\usepackage[T1]{fontenc}
\usepackage{lmodern}
\usepackage[english]{babel}
\usepackage[tbtags]{amsmath}
\usepackage{wasysym}
\usepackage{amssymb}
%%\usepackage[xdvi]{epsfig}
%\usepackage{ifthen}
%\usepackage{latexsym}
%\usepackage{theorem}
%\usepackage{varioref}
\usepackage{graphicx}
\usepackage{epic}
\usepackage{eepic}
%\usepackage{rotfloat}
\usepackage{multicol}
%\usepackage{wrapfig}
\usepackage{fancyhdr}
\usepackage{float}
%\usepackage{makeidx}
\usepackage{fancybox}
\usepackage{subfig}
\usepackage{endnotes}
\usepackage{natbib}				% REFERENCER FORMATERET SOM NATURVIDENSKAB

% % % % % % % % % % % % % % % % % % % FIGURER % % % % % % % % % % % % % % % % % % %
\newcommand{\includefigure}[4]{
\begin{figure}[htb]    
  \begin{center}
    \includegraphics[angle=0, scale=#4]{#1} 
    \caption{#2}                          
    \label{#3}           
  \end{center}
\end{figure}
}

% % % % % % % % % % % % % % % % % % % REFERENCER % % % % % % % % % % % % % % % % % % %
\renewcommand\notesname{Referencer}
\numberwithin{equation}{section}

%----Different font in captions----
\newcommand{\captionfonts}{\small}

\makeatletter  % Allow the use of @ in command names
\long\def\@makecaption#1#2{%
  \vskip\abovecaptionskip
  \sbox\@tempboxa{{\captionfonts #1: #2}}%
  \ifdim \wd\@tempboxa >\hsize
    {\captionfonts #1: #2\par}
  \else
    \hbox to\hsize{\hfil\box\@tempboxa\hfil}%
  \fi
  \vskip\belowcaptionskip}
\makeatother   % Cancel the effect of \makeatletter
%------------Define Keys-------------

%\evensidemargin = -0.5cm
%\oddsidemargin  =  2.8cm

\makeindex

\parskip        =    1ex
\parindent      =    0em
\baselineskip   =    2ex

%\pagestyle{empty}
%\underlineheadings

% % % % % % % % % % % % % % % % % % % FANCY SHIZZLE? % % % % % % % % % % % % % % % % % % %
\pagestyle{fancy} 
% with this we ensure that the chapter and section 
% headings are in lowercase. 
\renewcommand{\sectionmark}[1]{% 
\markright{\thesection\ #1}} 
\fancyhf{} % delete current header and footer 
\fancyhead[LE,RO]{\bfseries\thepage} 
\fancyhead[LO]{\bfseries\rightmark} 
\fancyhead[RE]{\bfseries\leftmark} 
\renewcommand{\headrulewidth}{0.5pt} 
\renewcommand{\footrulewidth}{0pt} 
\addtolength{\headheight}{0.5pt} % space for the rule 
\fancypagestyle{plain}{% 
\fancyhead{} % get rid of headers on plain pages 
\renewcommand{\headrulewidth}{0pt} % and the line 
}

\typeout{- SLUT ----------------------------------- Preamble.tex --}
% % % % % % % % % % % % % % % % % % % BEGIN DOCUMENT % % % % % % % % % % % % % % % % % % %
\begin{document}
\title{Bachelorprojekt\\Ikkekummutativ geometri og Standard Modellen}
\author{Aandreas Skielbo og Julie Hougaard}
\date{\today}
\maketitle
\pagenumbering{roman}

% % % % % % % % % % % % % % % % % % % ABSTRACT % % % % % % % % % % % % % % % % % % %
\begin{abstract}
I den foreliggende rapport skal vi undersøge hvilken effekt ikkekummunikativ rumtids geometri vil have på $Z^0$ henfald.
\end{abstract}

\clearpage
\tableofcontents
\clearpage

% % % % % % % % % % % % % % % % % % % HOVEDOPGAVE % % % % % % % % % % % % % % % % % % %

\pagenumbering{arabic}

Citation example.\cite{melic2005smn}

% % % % % % % % % % % % % % % % % % % INTRODUKTION % % % % % % % % % % % % % % % % % % %
\section{Introduction}
In describing fundamental physical phenomena, physicists make use of the concept of a particle representing a certain small-scale state (of the order $10^-15$ m) of the universe. Particles can interact with each other to annihilate or create new particles (or states). To account for this we use the concept of interactions. With these two concepts in hand we can go on and build mathematical frameworks based on experimental results and/or purely mathematical ideas, with the intend to extend our knowledge of the Universe and increase the predictive power and accuracy of the theories involved. Using this method physicists have been able to create an extensive theoretical framework with amazing predictive accuracy. This framework is commonly know as the Standard Model of particle physics, abbreviated SM.

Another accurate framework was developed in the beginning of the last century by german physicist Albert Einstein. This framework, describing gravity on large scales, is know as the General Theory of Relativity (GR). Together with the SM these two theories represent our best current knowledge\footnote{Henceforth when we write knowledge the reader may assume this is equivalent to the information available from being able to predict the time evolution of physical systems or states.} of the physical Universe at the very large and very small scales.

A lot of effort as been put in to unifying the encompassing physical theory of the large GR, with the theory of the s and the SM


% % % % % % % % % % % % % % % % % % % TEORI % % % % % % % % % % % % % % % % % % %
\include{theory}

% % % % % % % % % % % % % % % % % % % CALCULATIONS % % % % % % % % % % % % % % % % % % %


% % % % % % % % % % % % % % % % % % % RESULTS % % % % % % % % % % % % % % % % % % %


% % % % % % % % % % % % % % % % % % % CONCLUSION % % % % % % % % % % % % % % % % % % %

\clearpage

% % % % % % % % % % % % % % % % % % % BIBTEX % % % % % % % % % % % % % % % % % % %

\bibliographystyle{unsrt}
\bibliography{ncg_citations}	% Navn paa BibTex -filen

\clearpage

\pagenumbering{alph}

% % % % % % % % % % % % % % % % % % % APPENDICES % % % % % % % % % % % % % % % % % % %

\section{Appendix A}

\clearpage

\section{Appendix B}


\clearpage

\section{Appendix C}

\clearpage

\section{Appendix D}

% % % % % % % % % % % % % % % % % % % SLUT PRUT % % % % % % % % % % % % % % % % % % %

\end{document}