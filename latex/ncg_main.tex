\typeout{- START ----------------------------------- Preamble.tex --}
\documentclass[11pt,a4paper,titlepage]{article}
%\usepackage[danish]{babel}
\usepackage[T1]{fontenc}
\usepackage{lmodern}
\usepackage[english]{babel}
\usepackage[tbtags]{amsmath}
\usepackage{wasysym}
\usepackage{amssymb}
%%\usepackage[xdvi]{epsfig}
%\usepackage{ifthen}
%\usepackage{latexsym}
%\usepackage{theorem}
%\usepackage{varioref}
\usepackage{graphicx}
\usepackage{epic}
\usepackage{eepic}
%\usepackage{rotfloat}
\usepackage{multicol}
%\usepackage{wrapfig}
\usepackage{fancyhdr}
\usepackage{float}
%\usepackage{makeidx}
\usepackage{fancybox}
\usepackage{subfig}
\usepackage{endnotes}
\usepackage{natbib}				% REFERENCER FORMATERET SOM NATURVIDENSKAB

% % % % % % % % % % % % % % % % % % % FIGURER % % % % % % % % % % % % % % % % % % %
\newcommand{\includefigure}[4]{
\begin{figure}[htb]    
  \begin{center}
    \includegraphics[angle=0, scale=#4]{#1} 
    \caption{#2}                          
    \label{#3}           
  \end{center}
\end{figure}
}

% % % % % % % % % % % % % % % % % % % REFERENCER % % % % % % % % % % % % % % % % % % %
\renewcommand\notesname{Referencer}
\numberwithin{equation}{section}

%----Different font in captions----
\newcommand{\captionfonts}{\small}

\makeatletter  % Allow the use of @ in command names
\long\def\@makecaption#1#2{%
  \vskip\abovecaptionskip
  \sbox\@tempboxa{{\captionfonts #1: #2}}%
  \ifdim \wd\@tempboxa >\hsize
    {\captionfonts #1: #2\par}
  \else
    \hbox to\hsize{\hfil\box\@tempboxa\hfil}%
  \fi
  \vskip\belowcaptionskip}
\makeatother   % Cancel the effect of \makeatletter
%------------Define Keys-------------

%\evensidemargin = -0.5cm
%\oddsidemargin  =  2.8cm

\makeindex

\parskip        =    1ex
\parindent      =    0em
\baselineskip   =    2ex

%\pagestyle{empty}
%\underlineheadings

% % % % % % % % % % % % % % % % % % % FANCY SHIZZLE? % % % % % % % % % % % % % % % % % % %
\pagestyle{fancy} 
% with this we ensure that the chapter and section 
% headings are in lowercase. 
\renewcommand{\sectionmark}[1]{% 
\markright{\thesection\ #1}} 
\fancyhf{} % delete current header and footer 
\fancyhead[LE,RO]{\bfseries\thepage} 
\fancyhead[LO]{\bfseries\rightmark} 
\fancyhead[RE]{\bfseries\leftmark} 
\renewcommand{\headrulewidth}{0.5pt} 
\renewcommand{\footrulewidth}{0pt} 
\addtolength{\headheight}{0.5pt} % space for the rule 
\fancypagestyle{plain}{% 
\fancyhead{} % get rid of headers on plain pages 
\renewcommand{\headrulewidth}{0pt} % and the line 
}

\typeout{- SLUT ----------------------------------- Preamble.tex --}
% % % % % % % % % % % % % % % % % % % BEGIN DOCUMENT % % % % % % % % % % % % % % % % % % %
\begin{document}
\title{Bachelorprojekt\\Ikkekummutativ geometri og Standard Modellen}
\author{Aandreas Skielbo og Julie Hougaard}
\date{\today}
\maketitle
\pagenumbering{roman}

% % % % % % % % % % % % % % % % % % % ABSTRACT % % % % % % % % % % % % % % % % % % %
\begin{abstract}
I den foreliggende rapport skal vi undersøge hvilken effekt ikkekummunikativ rumtids geometri vil have på $Z^0$ henfald.
\end{abstract}

\clearpage
\tableofcontents
\clearpage

% % % % % % % % % % % % % % % % % % % HOVEDOPGAVE % % % % % % % % % % % % % % % % % % %

\pagenumbering{arabic}

Citation example.\cite{melic2005smn}

% % % % % % % % % % % % % % % % % % % INTRODUKTION % % % % % % % % % % % % % % % % % % %
\section{Introduction (In English)}
In describing fundamental physical phenomena, physicists make use of the concept of a particle representing a certain small-scale state (of the order $10^{-15}$ m) of the universe. Particles can interact with each other annihilating or creating new particles (or states). To account for this we use another concept, namely that of interactions. With these two concepts in hand we can go on and build mathematical frameworks based on experimental results and/or purely mathematical ideas, with the intend to extend our knowledge of the Universe and increase the predictive power and accuracy of the theories involved. Using this method physicists have been able to create an extensive theoretical framework of amazing predictive accuracy. This framework, know as the Standard Model (SM) of particle physics, is written in the mathematical language of Quantum Field Theory (QFT). In the context of QFT particles are thought of as excitations in underlying fields propagating through space-time.

Another accurate framework was developed in the beginning of the last century by german physicist Albert Einstein. This framework, describing gravity on large scales, is know as the General Theory of Relativity (GR). Together with the SM these two theories represent our best current knowledge of the physical Universe at the very large and very small scales.\footnote{Henceforth when we write knowledge the reader may assume this is equivalent to the information available from being able to predict the time evolution of physical systems or states.}

These two, although very useful in their own domains, are mutually incompatible. The problem arises in the definition of the fabric on with events, or processes occur. This fabric is made up of the 3 partial dimensions and 1 time-like dimension together forming a 4-dimensional manifold called spacetime.
In GR spacetime is thought of as a grid with infinitely many points, whilst in QFT (and therefore the SM) spacetime is described as a multitude of fluctuating fields resulting in a net polarization of the vacuum. In this realm points in space are no longer well defined as otherwise required by GR.

A lot of effort as been put in to unifying the encompassing physical theory of the Universe on the very largest scales GR, with the theory of Universe at the smallest scales, the SM. So far no one have been successful, but a lot of interesting theoretical candidates exists. In the following report/paper/article [HVAD ER DET??] we will concentrate on one of these candidates, namely the extension of the SM by Non-Commutative Geometry (NCG). We will see that this theory predicts new vertices between particles that are illegal in the normal context of the SM. These vertices arise from interactions between the neutral gauge boson Z$^0$, responsible for weak-force decays, and the charged gauge bosons the gluons ($g$), responsible for strong-force decays.

In the report we will also analyse the particular conditions necessary for the interactions to be observed in experiments such as LEP and LHC. Furthermore will explore the different constraints that current experimental data puts on the detailed structure of this particular NCG-extension of the SM.

\section{Introduction (In Danish)}

% % % % % % % % % % % % % % % % % % % TEORI % % % % % % % % % % % % % % % % % % %
\section{Theory}
The particles and interactions of the SM are described in the language of quantum mechanics and quantum field theory, which have a natural way of unifying the mathematical concepts in a concise way.

The particles are divided into two main groups; fermions and bosons. Where fermions are defined as having half-integral spin and are described by Fermi-Dirac statistics, these include the leptons and quarks. Fermions constitutes all known matter and as such they are sometimes described as matter particles. Bosons on the other hand have zero or integral spin and are described by Bose-Einstein statistics. Some of these, namely the gauge bosons, are responsible for the weak, strong and electromagnetic interactions. Therefore the gauge bosons are often called the force-carriers of their respective interactions. It may be appropriate to note that the formalism of quantum mechanics makes no clear distinction between the concepts of matter and force -particles.

[FIGUR MED PARTIKLER, den fra wiki]

\subsection{Gauge transformations}

\subsection{Gauge groups}
The gauge boson mediating the electromagnetic force is the photon ($\gamma$). The theory is derived from the U(1) gauge group, which is just the group of phase rotations.

\begin{equation}
    \psi \rightarrow e^{i\alpha} \psi
\end{equation}

The weak interaction is derived from the SU(2) group of unitary matrices with determinant 1. The gauge boson associated with this group are the $W^+$, $W^-$ and the $Z^0$ bosons.

Combining these two theories Weinberg and Salam [CITATION] arrived at what is called Quantum Electrodynamics (QED) which is described by the gauge group U(1) $\times$ SU(2).

The gauge bosons mediating the strong interaction are characterized by the SU(3) gauge group. The generators of which are the 8 Gell-Mann matrices giving rise to 8 gauge bosons of the strong interaction know as gluons, each having a property called color. Because of this color feature the gauge theory of the strong interaction is called Quantum Chromodynamics (QCD).

Combining QED and QCD into U(1) $\times$ SU(2) $\times$ SU(3) we arrive at what is know as the Standard Model (SM) of particle physics.

But one important feature is still missing.

\subsection{Gravity}
General Relativity (GR) is the theory describing


So far there have been no successful unification of the SM and General Relativity (GR), which is the theory describing all know macroscopic

% % % % % % % % % % % % % % % % % % % CALCULATIONS % % % % % % % % % % % % % % % % % % %


% % % % % % % % % % % % % % % % % % % RESULTS % % % % % % % % % % % % % % % % % % %


% % % % % % % % % % % % % % % % % % % CONCLUSION % % % % % % % % % % % % % % % % % % %

\clearpage

% % % % % % % % % % % % % % % % % % % BIBTEX % % % % % % % % % % % % % % % % % % %

\bibliographystyle{unsrt}
\bibliography{ncg_citations}	% Navn paa BibTex -filen

\clearpage

\pagenumbering{alph}

% % % % % % % % % % % % % % % % % % % APPENDICES % % % % % % % % % % % % % % % % % % %

\section{Appendix A}

\clearpage

\section{Appendix B}


\clearpage

\section{Appendix C}

\clearpage

\section{Appendix D}

% % % % % % % % % % % % % % % % % % % SLUT PRUT % % % % % % % % % % % % % % % % % % %

\end{document}