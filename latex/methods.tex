\section{Methods}

\subsection{The process of interest $(pp \rightarrow Z^0 \rightarrow \mu \bar\mu)$}
We have decided to look at a process frequently occurring at the LCH, namely muon pair-production, in which a muon anti-muon pair is produced from the collision of two protons. In the ordinary SM this process can happen in the following two ways; a quark anti-quark pair annihilating to create either a photon or a Z boson decaying into the muon-pair. These two processes are illustrated by a Feynman diagram in Figure \ref{fig:feyn:parton_qq}. In NCG we have the extra vertex $Zgg$ which allows us to draw a diagram like that in Figure \ref{fig:feyn:parton_gg}. In this case we have taken out two gluons from the proton which collide to produce a Z boson decaying into a muon-pair. Our strategy is to calculate the total cross section for muon pair-production at the LHC
\begin{equation}
	\sigma_{pp \rightarrow \gamma/ Z \rightarrow \mu \bar \mu} = |A_{q \bar q \rightarrow \gamma} + A_{q \bar q \rightarrow Z} + A_{gg \rightarrow Z}|^{2},
\end{equation}

where $A = \sigma P^{pp}$ is the experimental amplitude, $\sigma$ is the cross section at calculated by Fermi's Golden Rule \eqref{eq:goldenrule} and $P^{pp}$ is the parton distribution function. We have chosen to ignore the contribution coming from $gg \rightarrow \gamma$ because this term will make the calculations much more complex, although this is an interesting thing to look into in future studies. In Figure \ref{fig:feyn:parton_qq} and \ref{fig:feyn:parton_gg} we have drawn the two processes we will be looking at. While the full quantum field theoretical calculations for these processes are too advanced, for us to make use of, there exists several numerical application that can be used to gather the results we need for further analysis. We calculate all SM amplitudes directly in CompHEP, and derive the NCG contribution using ROOT and CompHEP histograms.

\begin{figure}[htp]
	\centering
	\begin{minipage}[b]{0.475\linewidth} 
    \centering
	  \begin{picture}(215,157) (22,2)
  \SetWidth{1.0}
  \SetColor{Black}
  \Line[arrow,arrowpos=0.5,arrowlength=5,arrowwidth=2,arrowinset=0.2,flip](163,80)(201,109)
  \Line[arrow,arrowpos=0.5,arrowlength=5,arrowwidth=2,arrowinset=0.2](163,80)(201,51)
  \Text(100,104)[lb]{\Large{\Black{$q$}}}
  \Text(100,54)[lb]{\Large{\Black{$\bar q$}}}
  \Text(202,116)[lb]{\Large{\Black{$\bar \mu$}}}
  \Text(202,42)[lb]{\Large{\Black{$\mu$}}}
  \Text(133,91)[lb]{\Large{\Black{$Z^0$}}}
  \Line[arrow,arrowpos=0.5,arrowlength=5,arrowwidth=2,arrowinset=0.2](77,109)(115,80)
  \Line[arrow,arrowpos=0.5,arrowlength=5,arrowwidth=2,arrowinset=0.2,flip](77,51)(115,80)
  \SetWidth{0.0}
  \Vertex(74,115){8.485}
  \Vertex(73,45){8.485}
  \SetWidth{1.0}
  \Line[dash,dashsize=6,arrow,arrowpos=0.5,arrowlength=5,arrowwidth=2,arrowinset=0.2,flip](65,115)(19,114)
  \Line[dash,dashsize=6,arrow,arrowpos=0.5,arrowlength=5,arrowwidth=2,arrowinset=0.2,flip](65,45)(19,45)
  \Line[arrow,arrowpos=0.5,arrowlength=5,arrowwidth=2,arrowinset=0.2,flip](116,3)(79,40)
  %\Line[arrow,arrowpos=0.5,arrowlength=5,arrowwidth=2,arrowinset=0.2](75,38)(110,3)  
  \Line[arrow,arrowpos=0.5,arrowlength=5,arrowwidth=2,arrowinset=0.2](80,44)(114,11)
  \Line[arrow,arrowpos=0.5,arrowlength=5,arrowwidth=2,arrowinset=0.2,flip](115,157)(77,119)
  %\Line[arrow,arrowpos=0.5,arrowlength=5,arrowwidth=2,arrowinset=0.2](81,116)(115,150)
  \Line[arrow,arrowpos=0.5,arrowlength=5,arrowwidth=2,arrowinset=0.2](74,123)(109,158)
  \Text(65,25)[lb]{\Large{\Black{$p$}}}
  \Text(64,126)[lb]{\Large{\Black{$p$}}}
  \Photon(116,80)(164,81){4.5}{4}
\end{picture}
	  \caption{Allowed in the SM; scattering between two protons with two quarks colliding and interacting to create a $Z^0$ boson decaying into a muon-pair.} \label{fig:feyn:parton_qq}
	\end{minipage}
	\hspace{0.5cm}
	\begin{minipage}[b]{0.475\linewidth} 
    \centering
	  \begin{picture}(333,261) (38,3)
    \SetWidth{1.0}
    \SetColor{Black}
    \Line[arrow,arrowpos=0.5,arrowlength=5,arrowwidth=2,arrowinset=0.2,flip](272,133)(336,181)
    \Line[arrow,arrowpos=0.5,arrowlength=5,arrowwidth=2,arrowinset=0.2](272,133)(336,85)
    \Text(167,172)[lb]{\Large{\Black{$g$}}}
    \Text(167,90)[lb]{\Large{\Black{$g$}}}
    \Text(336,193)[lb]{\Large{\Black{$\bar \mu$}}}
    \Text(336,69)[lb]{\Large{\Black{$\mu$}}}
    \Text(222,151)[lb]{\Large{\Black{$Z^0 / \gamma$}}}
    \Gluon(128,181)(192,133){7.5}{6}
    \Gluon(128,85)(192,133){7.5}{6}
    \Vertex(124,191){13.342}
    \Vertex(121,74){13.342}
    \Line[dash,dashsize=10,arrow,arrowpos=0.5,arrowlength=5,arrowwidth=2,arrowinset=0.2,flip](109,191)(32,190)
    \Line[dash,dashsize=10,arrow,arrowpos=0.5,arrowlength=5,arrowwidth=2,arrowinset=0.2,flip](109,74)(33,74)
    \Line[arrow,arrowpos=0.5,arrowlength=5,arrowwidth=2,arrowinset=0.2,flip](193,4)(132,65)
    \Line[arrow,arrowpos=0.5,arrowlength=5,arrowwidth=2,arrowinset=0.2](125,62)(183,4)
    \Line[arrow,arrowpos=0.5,arrowlength=5,arrowwidth=2,arrowinset=0.2](133,73)(190,18)
    \Line[arrow,arrowpos=0.5,arrowlength=5,arrowwidth=2,arrowinset=0.2,flip](192,261)(129,198)
    \Line[arrow,arrowpos=0.5,arrowlength=5,arrowwidth=2,arrowinset=0.2](135,193)(191,249)
    \Line[arrow,arrowpos=0.5,arrowlength=5,arrowwidth=2,arrowinset=0.2](124,204)(182,263)
    \Text(108,48)[lb]{\Large{\Black{$p$}}}
    \Text(107,210)[lb]{\Large{\Black{$p$}}}
    \Photon(193,133)(273,134){7.5}{4}
\end{picture}
	  \caption{Forbidden in the SM; scattering between two protons with two gluons colliding and interacting to create a $Z^0$ boson decaying into a muon-pair.} \label{fig:feyn:parton_gg}%  This process is strictly forbidden in ordinary SM, but is allowed in the modifications made to the SM by including NCG.
	\end{minipage}
\end{figure}


\subsection{Calculating cross sections in CompHEP}

CompHEP is a numerical application for calculating cross sections for multi-particle final states in elementary particle decays and collision events, in the lowest order of perturbation theory. CompeHEP starts from the level of Feynman rules for a gauge model Lagrangian. There are several built-in models, for instance the SM in Unitary and 't-Hooft - Feynman gauge, and some MSSM models. You can also create a new model of your own, or generate it with another program called LanHEP. In LanHEP one simply inputs the Lagrangian terms for the theory you want to study and the program derives matrix elements for direct use in CompHEP. Then CompHEP calculates symbolically the matrix element for any of the processes chosen. This was what we tried to do initially, but we ran into problems stemming from the nature of NCG.

Because LanHEP uses a lot of algorithms based on Lorentz invariance, one of the limitations is that you cannot introduce any Lorentz-invariant terms. This causes problems if you want to include a term \eqref{eq:zggterm}, where we have an anti-symmetric tensor of the type $\theta^{\mu\nu}$, given in \eqref{eq:ncgtheta}. The antisymmetry between the tensor components give rise to distinctions between directions in space, something that is in clear violation of Lorentz invariance, required by LanHEP. Even if we were to ignore the field strength tensor in \eqref{eq:ncgtheta}, and only plug in scalar $\theta$, CompHEP will not let us because of the-non conformity of the SU(2) structure of $Z^0$ and the SU(3) structure of the gluons.

These observations led us to consider different methods by which to study NCG using CompHEP.

\subsubsection{Finding the appropriate form for $\theta_\textrm{NCG}$}
Instead of trying to implement the full tensor expression for the NCG contribution $\theta$ \eqref{eq:ncgtheta} we approximate by an effective zero-point interaction. The scattering amplitude in lowest order perturbation theory for the tree-level interaction in Figure \ref{fig:feyn:parton_gg} is given in natural units by
\begin{equation}
	\mathcal{M}(\mathbf{q}) = -\frac{g_Z g_g }{|\mathbf{q}|^2 + \Lambda^2},
\end{equation}
where $g_g$ is associated with the vertex $Zgg$ and $g_Z$ with the vertex $Z\mu\mu$, $\Lambda$ is related to $\theta$ by \eqref{eq:ncgtheta}. Making the low energy approximation that $|\mathbf{q}|^2 << \Lambda^2$ we get
\begin{equation} \label{eq:scatteringamplitude}
	\mathcal{M} = -\frac{g_Z g_g}{\Lambda^2} = -G_{Zgg},
\end{equation}
where $G_Z$ is the effective low-energy constant.\footnote{See section 9.1 in \cite{martin1998pp}.} In the case that we want to calculate the cross section for the process in Figure \ref{fig:feyn:parton_gg} we make use of Fermi's Golden Rule \eqref{eq:goldenrule}
\begin{equation}
	\hat \sigma_{Zgg} \propto |\mathcal{M}|^2 = G_{Zgg}^2.
\end{equation}
Now $G_Z$ has the unit of [E$^{-4}$] but we want the cross section to be of dimension [E$^{-2}$]. The only dimensional parameter we have available is the CM energy $|\mathbf{q}|^2 = s$, that gives us
\begin{equation}
	\hat \sigma_{Zgg} \approx G_{Zgg}^2 s = \frac{g_Z^2 g_g^2}{\Lambda^4}s.
\end{equation}
Looking at the vertices we have to figure out what the couplings $g_Z$ and $g_g$ should be. The first one, $g_Z$, is related to a regular weak interaction, and we have looked it up to be \cite{bettini2008iep}
\begin{equation}
	g_Z = \frac{\sqrt{4\pi\alpha}}{\sin{\theta_W}\cos{\theta_W}} c_Z,
\end{equation}
where $c_Z$ is a so-called Z charge factor and $\alpha = 1/128$ is the electromagnetic coupling calculated on the Z-mass. In our case, with $\bar \mu_R + \mu_L$ in the final state, $c_Z$ is given by $c_Z = \pm (1/2 - (2/3)\sin^2{\theta_W})$. The other coupling $g_g$ we derive from the width of the $Z \rightarrow gg$ decay given by \cite{behr2003dnc}
\begin{equation}
	\frac{g_g^2}{\Lambda^4} \propto \Gamma_{Z \rightarrow gg} = \frac{8}{12} \alpha M_Z^5 \sin^2{2\theta_W} K_{Zgg}^2 \frac{1}{\Lambda^4}.
\end{equation}
Again we make use of dimensional analysis. As we can see from \eqref{eq:scatteringamplitude}, $G_{Zgg}$ should have the dimension of [$E^{-2}$]. To achieve this we divide by $M_Z^5$ to get
\begin{equation}
	g_g^2 = \frac{8}{12} \alpha \sin^2{2\theta_W} K_{Zgg}^2.
\end{equation}
Now we can write the complete cross section for $gg \rightarrow Z \rightarrow \mu \bar \mu$
\begin{equation}
	\hat \sigma_{gg \rightarrow Z \rightarrow \mu \bar \mu} = \frac{32\pi}{12} \frac{\sin^2{2\theta_W}}{\sin^2{\theta_W}\cos^2{\theta_W}} \alpha^2 c_Z^2 K_{Zgg}^2 \frac{s}{\Lambda^4} .
\end{equation}

\subsubsection{Deriving the cross section from CompHEP histograms}
Instead of implementing the NCG Lagrangian terms directly into CompHEP we examine the different amplitudes in the total cross section, including NCG contributions.

\begin{align}
\sigma_{pp \rightarrow \gamma/ Z \rightarrow \mu \bar \mu} &= |A_{q \bar q \rightarrow \gamma} + A_{q \bar q \rightarrow Z} + A_{gg \rightarrow Z}|^{2} \nonumber \\
&= |A_{q \bar q \rightarrow \gamma}|^{2} + |A_{q \bar q \rightarrow Z}|^ {2} + |A_{gg \rightarrow Z}|^{2} + |A_{q \bar q \rightarrow Z }A_{qq \rightarrow \gamma}|.
\end{align}

The last term is the interference part between $\gamma$ and Z, the second last is the cross section coming from NCG, that we are interested in. The two first terms we already know

\begin{align}
A_{q \bar q \rightarrow Z} &= \sigma_{q \bar q \rightarrow Z} (\mathcal{M}_{q \bar q}) \otimes \textrm{P}_{pp}(\mathcal{M}_{q \bar q}) \nonumber \\ 
A_{q \bar q \rightarrow \gamma} &= \sigma_{q \bar q \rightarrow  \gamma}( \mathcal{M}_{q \bar q}) \otimes \textrm{P}_{pp}(\mathcal{M}_{q \bar q}).
\end{align}

Here the symbol $\otimes$ means that we multiply distributions, bin by bin, $\mathcal{M}$ is the invariant mass for the two particles in the final state and P$_{pp}$ is the PDF. The idea is now to derive the last term, the amplitude of two gluons coupling to Z. Lets therefore take a look at the terms in the total muon production cross section that includes virtual $Z$ bosons.

\begin{equation}
\hat \sigma_{q \bar q/gg \rightarrow Z} = |A_{q \bar q \rightarrow Z} + A_{gg \rightarrow Z}|^{2}=|A_{q \bar q \rightarrow Z}|^{2}+|A_{gg \rightarrow Z}|^{2}.
\end{equation}

Notice that we avoid interference terms by insisting on distinguishable initial and final states. Now we insert the form of $\theta$ derived in the previous section, then we have
% trick is to write write $A_{gg \rightarrow Z}$ as a percentage of $A_{q \bar q \rightarrow Z}$

\begin{equation}
	\sigma_{gg\rightarrow Z} (\mathcal{M}_{gg}) \otimes \textrm{P}_{pp}(\mathcal{M}_{gg})=\sigma_{q \bar q\rightarrow Z}(\mathcal{M}_{q \bar q}) \otimes \theta_{NCG}\textrm{P}_{pp}(\mathcal{M}_{gg}).
\end{equation}
This leads us to an expression for the total cross section

\begin{align} \label{eq:qqggztotal}
\sigma_{q \bar q/gg \rightarrow Z} &= \sigma_{q \bar q \rightarrow Z} (\mathcal{M}_{q \bar q}) \otimes \textrm{P}_{pp}(\mathcal{M}_{q \bar q})+ \sigma_{gg\rightarrow Z} (\mathcal{M}_{gg}) \otimes \textrm{P}_{pp}(\mathcal{M}_{gg})\nonumber \\ 
&= \sigma_{q \bar q\rightarrow Z}(\mathcal{M}_{q \bar q}) \otimes (\textrm{P}_{pp}(\mathcal{M}_{q \bar q})+\theta_{NCG}^{\Lambda}\textrm{P}_{pp}(\mathcal{M}_{gg}))\nonumber \\ 
&= \sigma_{q \bar q\rightarrow Z}(\mathcal{M}_{q \bar q}) \otimes \textrm{P}_{pp}(\mathcal{M}_{q \bar q}) \left ( 1+\theta_{NCG}^{\Lambda}\frac{\textrm{P}_{pp}(\mathcal{M}_{gg})}{\textrm{P}_{pp}(\mathcal{M}_{q \bar q}} \right ).
\end{align}

\subsubsection{Finding the ratio ${\mathcal{L}_{pp}^{gg}}/{\mathcal{L}_{pp}^{q \bar q}}$}
To find the ratio between the two luminosity functions we made the distributions of two almost similar interactions with comphep. The two interactions was  $gg \rightarrow g \rightarrow q \bar q$ and $q \bar q \rightarrow g \rightarrow q \bar q$, se figures \ref{fig:feyn:parton_qq} and \ref{fig:feyn:parton_gg}, with  $\sigma^{pp}_{gg \rightarrow g \rightarrow q \bar q}=\hat \sigma_{gg \rightarrow g \rightarrow q \bar q} \otimes \mathcal{L}_{pp}^{gg}$ and $\sigma^{pp}_{q \bar q \rightarrow g \rightarrow q \bar q}= \hat \sigma_{q \bar q \rightarrow g \rightarrow q \bar q} \otimes \mathcal{L}_{pp}^{q \bar q}$. If $\hat \sigma_{gg \rightarrow g \rightarrow q \bar q} = \hat \sigma_{q \bar q \rightarrow g \rightarrow q \bar q}$ up to a constant then $\sigma^{pp}_{gg \rightarrow g \rightarrow q \bar q}/\sigma^{pp}_{q \bar q \rightarrow g \rightarrow q \bar q}$ equals $\mathcal{L}_{pp}^{gg}/\mathcal{L}_{pp}^{q \bar q}$ up to the same constant. The choice of interactions is very smart since they are strong interactions, and therefore the coupling constant being $\alpha_{s}$ for both and therefor dissapears. All we then have left is the ratio between the different color factors.
As mentioned in section \ref{sec:su2andsu3} and \ref{sec:particles} quarks 3 different colors and gluons have 8. Therefore we have\footnote{See p. 569 in  \cite{peskin1993iqf}.}:
\begin{align}
	CF_{gg \rightarrow g \rightarrow q \bar q}&=\frac{1}{8} \cdot \frac{1}{8} \cdot 8=\frac{8}{64}\\
	CF_{q \bar q \rightarrow g \rightarrow q \bar q}&=\frac{1}{3} \cdot \frac{1}{3} \cdot 8=\frac{1}{9}.
\end{align}

This gives us the ratio

\begin{align}
	\frac{\hat \sigma_{gg \rightarrow g \rightarrow q \bar q}}{\hat \sigma_{q \bar q \rightarrow g \rightarrow q \bar q}} &=\frac{2 \cdot \alpha_{s} \cdot \frac{8}{64} \cdot C^{2}(r)}{2 \cdot \alpha_{s} \cdot \frac{8}{9} \cdot C^{2}(r)}=\frac{9}{64},
\end{align}
where $C(r) = \frac{1}{2}$\cite{peskin1993iqf} for the fundamental representation of SU(3) and the factor of two accounts for the symmetry of swapping the two particles in the initial state.


Now we are actually ready to create the cross sections, needed for analysis, with CompHEP. We collide protons with beam energy 7000 GeV and then looks at these different processes:

\begin{align}
gg &\rightarrow g \rightarrow u \bar u \nonumber \\ \nonumber
u \bar u &\rightarrow g \rightarrow d \bar d \\ \nonumber
q \bar q &\rightarrow Z \rightarrow \mu \bar \mu \\ \nonumber
q \bar q &\rightarrow \gamma \rightarrow \mu \bar \mu,
\end{align}
where $q = \{u,d,s,c,b\}$. The first two processes we use to calculate the PDF ratio in \eqref{eq:qqggztotal}, the rest we add together to get the total cross section for all possible collisions. The reason we're not just doing $q \bar q \rightarrow Z/ \gamma \rightarrow \mu \bar \mu$, which CompHEP is fully capable of, is that we need the partial cross section for processes involving only $Z$'s to be able to add the NCG contribution. To avoid singularities for $P_\textrm{T} \rightarrow 0$ we put in cuts in the transverse momentum of the out-coming particles (> 50 GeV) and also in their invariant mass ($20 \leq M \leq 1820$ GeV).

\subsection{Numerical tools, ROOT \& Mathematica}
We use two different programs for analysis and making plots. the first one is ROOT. ROOT is a program for data processing, developed at CERN. The package has many features. We mostly used it for calculating with and analyzing histogram distribution with we gaind from CompHEP. ROOT provides a data structure that is extremely powerful for fast access of huge amounts of data. This is very appropriate when making particle physics data analysis. ROOT is written in C++, and we're using version 5.23. For some analytical calculations and simple numerical plotting we have used Mathematica 6. For more information about ROOT see http://root.cern.ch.
