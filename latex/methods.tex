\section{Methods}

\subsection{The process of interest $(pp \rightarrow Z^0 \rightarrow \mu \bar\mu)$}
We have decided to look at a process frequently occurring at the LCH, namely muon pair-production, in which a muon anti-muon pair is produced from the collision of two protons. In the ordinary SM this process can happen in the following two ways; a quark anti-quark pair annihilating to create either a photon or a Z boson decaying into the muon-pair. These two processes are illustrated by a feynman diagram in Figure \ref{fig:feyn:pp-qq-z-mumu}. In NCG we have the extra vertex $Zgg$ which allows us to draw a diagram like that in Figure \ref{fig:feyn:pp-gg-z-mumu}. In this case we have taken out two gluons from the proton which collide to produce a Z boson decaying into a muon-pair. Our strategy is to calculate the total cross section for muon pair-production at the LHC
\begin{equation}
	\sigma_{pp \rightarrow \gamma/ Z \rightarrow \mu \bar \mu} = |A_{q \bar q \rightarrow \gamma} + A_{q \bar q \rightarrow Z} + A_{gg \rightarrow Z} + A_{gg \rightarrow \gamma} + [INTERFERENS!?]|^{2},
\end{equation}
where $A = \sigma P^{pp}$ is the experimental amplitude, $\sigma$ is the cross section at calculated by Fermi's Golden Rule \eqref{eq:goldenrule} and $P^{pp}$ is the parton distribution function. We hereby hope to be able to examine these results to set constraints on NCG by current, and someday future experimental data. While the full quantum field theoretical calculation is too advanced, for us to make use of, there exists several numerical application that can be used to gather the results we need for further analysis.

\begin{figure}[htp]
	\centering
	\begin{minipage}[b]{0.475\linewidth} \label{fig:feyn:parton1}
    \centering
	  \begin{picture}(215,157) (22,2)
  \SetWidth{1.0}
  \SetColor{Black}
  \Line[arrow,arrowpos=0.5,arrowlength=5,arrowwidth=2,arrowinset=0.2,flip](163,80)(201,109)
  \Line[arrow,arrowpos=0.5,arrowlength=5,arrowwidth=2,arrowinset=0.2](163,80)(201,51)
  \Text(100,104)[lb]{\Large{\Black{$q$}}}
  \Text(100,54)[lb]{\Large{\Black{$\bar q$}}}
  \Text(202,116)[lb]{\Large{\Black{$\bar \mu$}}}
  \Text(202,42)[lb]{\Large{\Black{$\mu$}}}
  \Text(133,91)[lb]{\Large{\Black{$Z^0$}}}
  \Line[arrow,arrowpos=0.5,arrowlength=5,arrowwidth=2,arrowinset=0.2](77,109)(115,80)
  \Line[arrow,arrowpos=0.5,arrowlength=5,arrowwidth=2,arrowinset=0.2,flip](77,51)(115,80)
  \SetWidth{0.0}
  \Vertex(74,115){8.485}
  \Vertex(73,45){8.485}
  \SetWidth{1.0}
  \Line[dash,dashsize=6,arrow,arrowpos=0.5,arrowlength=5,arrowwidth=2,arrowinset=0.2,flip](65,115)(19,114)
  \Line[dash,dashsize=6,arrow,arrowpos=0.5,arrowlength=5,arrowwidth=2,arrowinset=0.2,flip](65,45)(19,45)
  \Line[arrow,arrowpos=0.5,arrowlength=5,arrowwidth=2,arrowinset=0.2,flip](116,3)(79,40)
  %\Line[arrow,arrowpos=0.5,arrowlength=5,arrowwidth=2,arrowinset=0.2](75,38)(110,3)  
  \Line[arrow,arrowpos=0.5,arrowlength=5,arrowwidth=2,arrowinset=0.2](80,44)(114,11)
  \Line[arrow,arrowpos=0.5,arrowlength=5,arrowwidth=2,arrowinset=0.2,flip](115,157)(77,119)
  %\Line[arrow,arrowpos=0.5,arrowlength=5,arrowwidth=2,arrowinset=0.2](81,116)(115,150)
  \Line[arrow,arrowpos=0.5,arrowlength=5,arrowwidth=2,arrowinset=0.2](74,123)(109,158)
  \Text(65,25)[lb]{\Large{\Black{$p$}}}
  \Text(64,126)[lb]{\Large{\Black{$p$}}}
  \Photon(116,80)(164,81){4.5}{4}
\end{picture}
	  \caption{Allowed in the SM; scattering between two protons with two quarks colliding and interacting to create a $Z^0$ boson decaying into a muon-pair.} 
	  \label{fig:feyn:pp-qq-z-mumu}
	\end{minipage}
	\hspace{0.5cm}
	\begin{minipage}[b]{0.475\linewidth} \label{fig:feyn:parton2}
    \centering
	  \begin{picture}(333,261) (38,3)
    \SetWidth{1.0}
    \SetColor{Black}
    \Line[arrow,arrowpos=0.5,arrowlength=5,arrowwidth=2,arrowinset=0.2,flip](272,133)(336,181)
    \Line[arrow,arrowpos=0.5,arrowlength=5,arrowwidth=2,arrowinset=0.2](272,133)(336,85)
    \Text(167,172)[lb]{\Large{\Black{$g$}}}
    \Text(167,90)[lb]{\Large{\Black{$g$}}}
    \Text(336,193)[lb]{\Large{\Black{$\bar \mu$}}}
    \Text(336,69)[lb]{\Large{\Black{$\mu$}}}
    \Text(222,151)[lb]{\Large{\Black{$Z^0 / \gamma$}}}
    \Gluon(128,181)(192,133){7.5}{6}
    \Gluon(128,85)(192,133){7.5}{6}
    \Vertex(124,191){13.342}
    \Vertex(121,74){13.342}
    \Line[dash,dashsize=10,arrow,arrowpos=0.5,arrowlength=5,arrowwidth=2,arrowinset=0.2,flip](109,191)(32,190)
    \Line[dash,dashsize=10,arrow,arrowpos=0.5,arrowlength=5,arrowwidth=2,arrowinset=0.2,flip](109,74)(33,74)
    \Line[arrow,arrowpos=0.5,arrowlength=5,arrowwidth=2,arrowinset=0.2,flip](193,4)(132,65)
    \Line[arrow,arrowpos=0.5,arrowlength=5,arrowwidth=2,arrowinset=0.2](125,62)(183,4)
    \Line[arrow,arrowpos=0.5,arrowlength=5,arrowwidth=2,arrowinset=0.2](133,73)(190,18)
    \Line[arrow,arrowpos=0.5,arrowlength=5,arrowwidth=2,arrowinset=0.2,flip](192,261)(129,198)
    \Line[arrow,arrowpos=0.5,arrowlength=5,arrowwidth=2,arrowinset=0.2](135,193)(191,249)
    \Line[arrow,arrowpos=0.5,arrowlength=5,arrowwidth=2,arrowinset=0.2](124,204)(182,263)
    \Text(108,48)[lb]{\Large{\Black{$p$}}}
    \Text(107,210)[lb]{\Large{\Black{$p$}}}
    \Photon(193,133)(273,134){7.5}{4}
\end{picture}
	  \caption{Forbidden in the SM; scattering between two protons with two gluons colliding and interacting to create a $Z^0$ boson decaying into a muon-pair.} %  This process is strictly forbidden in ordinary SM, but is allowed in the modifications made to the SM by including NCG.
	  \label{fig:feyn:pp-gg-z-mumu}
	\end{minipage}
\end{figure}


\subsection{CompHEP and LanHEP}

CompHEP is a program for calculations of multi-particle final states in elementary particle decays and collision events in the lowest order of perturbation theory. CompeHEP starts from the level of Feynman rules for a gauge model Lagrangian. There are several built-in models, for instance the SM in Unitary and 't-Hooft - Feynman gauge, and some MSSM models, but you can also create a new model of your own or generate it with another program called LanHEP. In LanHEP one simply inputs the Lagrangian terms for the theory you want to study and the program derives matrix elements for direct use in CompHEP.  Then CompHEP calculates symbolically the matrix element for any of the processes chosen.

Because LanHEP uses a lot of algorithms based on Lorentz invariance one limitation in the program is that you cannot introduce any Lorentz-invariant terms. This causes problems if you want to include a term of the type \eqref{eq:zggterm} where we have an anti-symmetric tensor of the type $\theta^{\mu\nu}$ given in \eqref{eq:ncgtheta}. The antisymmetry between the tensor component give rise to distinctions between directions in space, something that is in clear violation of Lorentz invariance, required by LanHEP. These observations led us to consider different methods by which to study NCG using CompHep.

\subsubsection{Deriving the cross section from CompHEP histograms}

The idea is to calculate the NCG cross section for the new contribution gg $\rightarrow Z^0$ in the interaction pp $\rightarrow \mu \bar \mu$. To find this contribution we have to examine the different amplitudes in the total cross section.

\begin{align}
\sigma_{pp \rightarrow \gamma/ Z \rightarrow \mu \bar \mu} &= |A_{q \bar q \rightarrow \gamma} + A_{q \bar q \rightarrow Z} + A_{gg \rightarrow Z}|^{2} \nonumber \\
&= |A_{q \bar q \rightarrow \gamma}|^{2} + |A_{q \bar q \rightarrow Z}|^ {2} + |A_{gg \rightarrow Z}|^{2} + A_{q \bar q \rightarrow Z }A_{qq \rightarrow \gamma}.
\end{align}

The last term is the interference part between $\gamma$ and Z, the second last is the cross section coming from NCG, that we are interested in. The two first terms we already know

\begin{align}
A_{q \bar q \rightarrow Z} &= \sigma_{q \bar q \rightarrow Z} (\mathcal{M}_{q \bar q}) \oplus \textrm{P}_{pp}(\mathcal{M}_{q \bar q}) \nonumber \\ 
A_{q \bar q \rightarrow \gamma} &= \sigma_{q \bar q \rightarrow  \gamma}( \mathcal{M}_{q \bar q}) \oplus \textrm{P}_{pp}(\mathcal{M}_{q \bar q}).
\end{align}

Here the symbol $\oplus$ means that we multiply distributions, bin by bin individually, $\mathcal{M}$ is the invariant mass for the two particles in the final state and P$_{pp}$ is the parton luminosity function [DEN MÅ VI OGSÅ HELLERE LIGE DEFINERE]. Now our job is to calculate the last term, the amplitude of two gluons coupling to Z. Lets therefore take a look at the cross section for the Z-production.

\begin{equation}
\hat \sigma_{q \bar q/gg \rightarrow Z} = |A_{q \bar q \rightarrow Z} + A_{gg \rightarrow Z}|^{2}=|A_{q \bar q \rightarrow Z}|^{2}+|A_{gg \rightarrow Z}|^{2}.
\end{equation}

We notice that this is not the total Z-production since the interference part disappear. We can only create interference if we have the same initial and final state, here we have different initial states. If we could [MEN DET GØR VI JO, SÅ DET KAN VI VEL!?] write $A_{gg \rightarrow Z}$ as a percentage of $A_{q \bar q \rightarrow Z}$ then we would have

\begin{equation}
\sigma_{gg\rightarrow Z} (\mathcal{M}_{gg}) \oplus \textrm{P}_{pp}(\mathcal{M}_{gg})=\sigma_{q \bar q\rightarrow Z}(\mathcal{M}_{q \bar q}) \oplus \theta_{NCG}^{\Lambda}\textrm{P}_{pp}(\mathcal{M}_{gg})
\end{equation}

where $\theta_{NCG}$ is the NCG .... $\Lambda$ indicates the mass in the vertex gg $\rightarrow$ Z. With this in mind we can write the cross section as:

\begin{align}
\sigma_{q \bar q/gg \rightarrow Z} &= \sigma_{q \bar q \rightarrow Z} (\mathcal{M}_{q \bar q}) \oplus \textrm{P}_{pp}(\mathcal{M}_{q \bar q})+ \sigma_{gg\rightarrow Z} (\mathcal{M}_{gg}) \oplus \textrm{P}_{pp}(\mathcal{M}_{gg})\nonumber \\ 
&= \sigma_{q \bar q\rightarrow Z}(\mathcal{M}_{q \bar q}) \oplus (\textrm{P}_{pp}(\mathcal{M}_{q \bar q})+\theta_{NCG}^{\Lambda}\textrm{P}_{pp}(\mathcal{M}_{gg}))\nonumber \\ 
&= \sigma_{q \bar q\rightarrow Z}(\mathcal{M}_{q \bar q}) \oplus \textrm{P}_{pp}(\mathcal{M}_{q \bar q}) \left ( 1+\theta_{NCG}^{\Lambda}\frac{\textrm{P}_{pp}(\mathcal{M}_{gg})}{\textrm{P}_{pp}(\mathcal{M}_{q \bar q}} \right )
\end{align}

Well, all we now need to know is the ratio P$_{pp}(\mathcal{M}_{gg})$/P$_{pp}(\mathcal{M}_{q \bar q})$ since $ \sigma_{q \bar q\rightarrow Z}(\mathcal{M}_{q \bar q}) \oplus \textrm{P}_{pp}$ is what CompHEP gives us when we make the interaction: $u \bar u \rightarrow Z \rightarrow \mu \bar \mu $. If we can construct two processes: $A_{pp}^{A}=\sigma^{A} \oplus$ P$_{pp}^{gg}$ and $A_{pp}^{B}=\sigma^{B} \oplus$ P$_{pp}^{q \bar q}$, such that $A^{A}=A^{B}$, then $\sigma_{pp}^{A}/\sigma_{pp}^{B}$ equals  P$_{pp}(\mathcal{M}_{gg})$/P$_{pp}(\mathcal{M}_{q \bar q})$ up to a scaling factor.

The interactions we choose are A: $gg \rightarrow g \rightarrow q \bar q$ and B: $q \bar q \rightarrow g \rightarrow q \bar q$. The coupling constant being $\alpha_{s}$. If we take a glance at the Feynman diagrams in Figures \ref{fig:feyn:parton1} and \ref{fig:feyn:parton2} for the two interactions, we can simply find the different color factors by counting:\footnote{See p. 67 in  \cite{peskin1993iqf}.}
\begin{align}
	CF_{gg \rightarrow g \rightarrow q \bar q}&=\frac{1}{8} \cdot \frac{1}{8} \cdot 8=\frac{8}{64}\\
	%CF_{q \bar q \rightarrow \gamma \rightarrow q \bar q}&=\frac{1}{3} \cdot \frac{1}{3} \cdot 8=\frac{8}{9}\\
	CF_{q \bar q \rightarrow g \rightarrow q \bar q}&=\frac{1}{3} \cdot \frac{1}{3}=\frac{1}{9}.
\end{align}

This gives us the ratio

\begin{align}
	%\frac{\sigma_{gg \rightarrow g \rightarrow q \bar q}}{\sigma_{q \bar q \rightarrow \gamma \rightarrow q \bar q}} &=\frac{2 \cdot \alpha_{s} \cdot \frac{8}{64} \cdot C^{2}(r)}{2 \cdot \alpha_{qed} \cdot \frac{1}{9}}=\frac{9 \cdot  \alpha_{s} }{32 \cdot  \alpha_{qed} }\\ \nonumber \\
	\frac{\sigma_{gg \rightarrow g \rightarrow q \bar q}}{\sigma_{q \bar q \rightarrow g \rightarrow q \bar q}} &=\frac{2 \cdot \alpha_{s} \cdot \frac{8}{64} \cdot C^{2}(r)}{2 \cdot \alpha_{s} \cdot \frac{8}{9} \cdot C^{2}(r)}=\frac{9}{68},
\end{align}
where $C(r) = \frac{1}{2}$ for the fundamental representation of SU(3) and the factor of two accounts for the symmetry of swapping the two particles in the initial state.

Now we are actually ready to create the wanted cross sections with CompHEP. We collide two protons with beam energy 7000 GeV and then looks at different outcomes:

\begin{align}
gg \rightarrow g \rightarrow u \bar u \nonumber \\ \nonumber
u \bar u \rightarrow g \rightarrow d \bar d \\ \nonumber
u \bar u \rightarrow \gamma \rightarrow d \bar d \\ \nonumber
u \bar u \rightarrow Z \rightarrow \mu \bar \mu \\ \nonumber
u \bar u \rightarrow \gamma \rightarrow \mu \bar \mu 
\end{align}

We have made cuts in the transverse momentum of the out-coming particles (from 50 GeV) and also in their invariant mass (from 20 to 1820 GeV). 

\subsection{Calculating results using Root}
ROOT is a program for data processing, delveloped at CERN. The package has many features. We mostly used it for calculating with and analysing histogram distribution with we gaind from CompHEP. ROOT provides a data structure that is extremely powerful for fast access of huge amounts of data. This is very appropriate when making particle physics data analysis. It is written in C++.
