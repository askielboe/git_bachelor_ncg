\section{Methods}
The idea is to calculate the NCG cross section for the new contribution gg $\rightarrow Z^0$ in the interaction pp $\rightarrow \mu \bar \mu$. To find this contribution we have to examine the different amplitudes in the total cross section:

\begin{align}
\sigma_{pp \rightarrow \gamma/ Z \rightarrow \mu \bar \mu} &= |A_{q \bar q \rightarrow \gamma} + A_{q \bar q \rightarrow Z} + A_{gg \rightarrow Z}|^{2} \nonumber \\
&= |A_{q \bar q \rightarrow \gamma}|^{2} + |A_{q \bar q \rightarrow Z}|^ {2} + |A_{gg \rightarrow Z}|^{2} + A_{q \bar q \rightarrow Z }A_{qq \rightarrow \gamma},
\end{align}

The last term is the interference part between $\gamma$ and Z, the second last is the cross section coming from NCG, that we are interestet in. The two first terms we already know:

\begin{align}
A_{q \bar q \rightarrow Z} &= \sigma_{q \bar q \rightarrow Z} (\mathcal{M}_{q \bar q}) \oplus \textrm{P}_{pp}(\mathcal{M}_{q \bar q}) \nonumber \\ 
A_{q \bar q \rightarrow \gamma} &= \sigma_{q \bar q \rightarrow  \gamma}( \mathcal{M}_{q \bar q}) \oplus \textrm{P}_{pp}(\mathcal{M}_{q \bar q})
\end{align}

Here the symbol $\oplus$ means that we multiply distibutions, bin by bin individual, $\mathcal{M}$ is the invariant mass for the two particles in the final state and P$_{pp}$ is the parton luminosity function. Know our job is to find the last one, the amplitude of two gluons coupling to Z. Lets therefore take af look at the cross section for the Z-production.

\begin{equation}
\sigma_{q \bar q/gg \rightarrow Z} = |A_{q \bar q \rightarrow Z} + A_{gg \rightarrow Z}|^{2}=|A_{q \bar q \rightarrow Z}|^{2}+|A_{gg \rightarrow Z}|^{2}.
\end{equation}

We notice that this is not the total Z-production since the inteference part disappear. We can only create inteference if we have the same initial and final state, here we have different initial states. If we could write $A_{gg \rightarrow Z}$ as af percentage of $A_{q \bar q \rightarrow Z}$ then we would have

\begin{equation}
\sigma_{gg\rightarrow Z} (\mathcal{M}_{gg}) \oplus \textrm{P}_{pp}(\mathcal{M}_{gg})=\sigma_{q \bar q\rightarrow Z}(\mathcal{M}_{q \bar q}) \oplus \theta_{NCG}^{\Lambda}\textrm{P}_{pp}(\mathcal{M}_{gg})
\end{equation}

where $\theta_{NCG}$ is the NCG .... $\Lambda$ indicates the mass in the vertex gg $\rightarrow$ Z. With this in mind we can write the cross section as:

\begin{align}
\sigma_{q \bar q/gg \rightarrow Z} &= \sigma_{q \bar q \rightarrow Z} (\mathcal{M}_{q \bar q}) \oplus \textrm{P}_{pp}(\mathcal{M}_{q \bar q})+ \sigma_{gg\rightarrow Z} (\mathcal{M}_{gg}) \oplus \textrm{P}_{pp}(\mathcal{M}_{gg})\nonumber \\ 
&= \sigma_{q \bar q\rightarrow Z}(\mathcal{M}_{q \bar q}) \oplus (\textrm{P}_{pp}(\mathcal{M}_{q \bar q})+\theta_{NCG}^{\Lambda}\textrm{P}_{pp}(\mathcal{M}_{gg}))
\end{align}

Well, all we know need to know is the ratio P$_{pp}(\mathcal{M}_{gg})$/P$_{pp}(\mathcal{M}_{q \bar q})$. If we can construct two processes: $\sigma_{pp}^{A}=\sigma^{A} \oplus$ P$_{pp}^{gg}$ and $\sigma_{pp}^{B}=\sigma^{B} \oplus$ P$_{pp}^{q \bar q}$, such that $\sigma^{A}=\sigma^{B}$, then $\sigma_{pp}^{A}/\sigma_{pp}^{B}$ equals  P$_{pp}(\mathcal{M}_{gg})$/P$_{pp}(\mathcal{M}_{q \bar q})$ up to a scaling factor.

Since both the photons and gluons are massless they should have the same distribution or at least the distrubutions should be proportional. Thats why we choose interaction A to be $gg \rightarrow g \rightarrow q \bar q$ and interaction B to be $q \bar q \rightarrow \gamma \rightarrow q \bar q$. The coupling constants are respectively $\alpha_{s}$ and $\alpha_{qed}$\\


We could also choose B to be  $q \bar q \rightarrow g \rightarrow q \bar q$, and maby this would be even smarter because now A and B share $\alpha_{s}$. If we take a glance at the Feynmann diagrams for the three interactions we can simply by counting find the different color factors:
\begin{align}
	CF_{gg \rightarrow g \rightarrow q \bar q}&=\frac{1}{8} \cdot \frac{1}{8} \cdot 8=\frac{8}{64}\\
	CF_{q \bar q \rightarrow \gamma \rightarrow q \bar q}&=\frac{1}{3} \cdot \frac{1}{3} \cdot 8=\frac{8}{9}\\
	CF_{q \bar q \rightarrow g \rightarrow q \bar q}&=\frac{1}{3} \cdot \frac{1}{3}=\frac{1}{9}
\end{align}

This gives us the two ratios:

\begin{align}
	\frac{\sigma_{gg \rightarrow g \rightarrow q \bar q}}{\sigma_{q \bar q \rightarrow \gamma \rightarrow q \bar q}} &=\frac{2 \cdot \alpha_{s} \cdot \frac{8}{64} \cdot C^{2}(r)}{2 \cdot \alpha_{qed} \cdot \frac{1}{9}}=\frac{9 \cdot  \alpha_{s} }{32 \cdot  \alpha_{qed} }\\ \nonumber \\
	\frac{\sigma_{gg \rightarrow g \rightarrow q \bar q}}{\sigma_{q \bar q \rightarrow g \rightarrow q \bar q}} &=\frac{2 \cdot \alpha_{s} \cdot \frac{8}{64} \cdot C^{2}(r)}{2 \cdot \alpha_{s} \cdot \frac{8}{9} \cdot C^{2}(r)}=\frac{9}{68}
\end{align}

Where $C(r) = \frac{1}{2}$ for the fundamental representation og SU(3) and the factor of two accounts for the symmetry of swapping the two particles in the initial state.

Now we are actually ready to create the wanted cross sections with CompHEP. We collide two protons with beam energy 7000 GeV and then looks at diffferent outcomes:

\begin{align}
gg \rightarrow g \rightarrow u \bar u \nonumber \\ \nonumber
u \bar u \rightarrow g \rightarrow d \bar d \\ \nonumber
u \bar u \rightarrow \gamma \rightarrow d \bar d \\ \nonumber
u \bar u \rightarrow Z \rightarrow \mu \bar \mu \\ \nonumber
u \bar u \rightarrow \gamma \rightarrow \mu \bar \mu 
\end{align}

We have made cuts in transverse momentum (min 50 rad) and invariant mass (from 20 to 1820 GeV). 


