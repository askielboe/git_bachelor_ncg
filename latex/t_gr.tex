\subsection{General relativity}
General relativity is the theory of gravity developed by Einstein [CITATION] in 1915. You might have noticed that we have not mentioned gravity jet, and that is because the SM only is a theory of the electromagnetic, weak and strong interactions, it says nothing about gravity. GR describes gravity not by the concept of force, as Newton did, but as a property of the geometry of space and time. Another important concept to mention is the equivalence principle. This states that the laws of physics are the same to all inertial observers, independent on the observers velocity.

\subsubsection{Space-time}
Given that space and time are related, measurements in space can be related to measurements in time by the speed of light. The resulting space-time is thought of as a smooth and continuous manifold. In QFT the mathematical descriptions depends on particle fields embedded in the flat space-time of special relativity. In GR gravity is modelled as a curvature within space-time that changes as mass or energy moves. Therefore two bodies must impact the space-time curvature of each other. This phenomenon is called tidal force \footnote{Named after the tides wich is a result of the gravitational effects between the moon and the earth.}. You can imagine some kind of bending of the coordinatesystem around the masses. Therefore GR is incompatible with QFT. The tidal forces causes the gravitational effects not to be conservative and therefore the path-integral is dependent on the path taken. This means that we cannot presume that each point are "point-like", we must take each point with a given extension depending on the uncertainty principle \footnote{The Uncertainty Principle states that for a pair of conjugate variables such as position/momentum and energy/time, it is impossible to have a precisely determined value of each member of the pair at the same time.[CITATION].}. Therefor each point have a given precision. According to the unsertainty principle $\Delta E \Delta t \approx {h \over 2 \pi}$, energy and time are related in such a way that the conservation of energy can be violated, although for a very short moment. This allows a temporary change in the amount of energy in a piont in the space-time and thereby the cration of particle-antiparticle pairs of virtual particles. This phenomenon is called quantum fluctuations.


\subsubsection{Quantum gravity and the Graviton}
It is possible to describe gravity in the framework of quantum field theory like the other fundamental forces, such that the attractive force of gravity arises due to exchange of virtual gravitons, in the same way as the electromagnetic force arises from exchange of virtual photons. This reproduces general relativity in the classical limit. However, this approach fails at short distances of the order of the Planck length, where a more complete theory of quantum gravity, or a new approach to quantum mechanics, is required. Theories (strings, NCG) tries to solve this by defining a minimum length-scale to smear out quantum fluctuations in hope of making QG renormalizable.