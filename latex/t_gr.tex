\subsection{General relativity}
General relativity is the theory of gravity developed by Einstein in 1915. While the SM of particle physics is able to account for an amazing variety of electromagnetic, weak and strong interactions, it tells us nothing about gravity. GR describes gravity not by the concept of force, as Newton did, but as a property of the geometry of space and time. Another important concept to mention is the equivalence principle. This states that the laws of physics are the same to all inertial observers, independent on the observers velocity.

\subsubsection{Space-time}
Given that space and time are related, measurements in space can be related to measurements in time by the speed of light. The resulting space-time is thought of as a smooth and continuous manifold. In QFT the mathematical descriptions depends on fields embedded in the flat Minkowski space-time of special relativity. In GR gravity is modeled as a curvature within space-time that changes as mass or energy moves. Thus two bodies have an influence on the the space-time curvature in the vicinity of each other. One phenomenon resulting from this is the tidal force.\footnote{Tidal forces are also responsible for tides on Earth, as a result of the varying distance between Earth and the Moon, and the fact that the gravitational attraction between the two bodies varies across their diameter.} You can imagine some kind of bending of the coordinate system around the masses. Therefore GR is incompatible with QFT. The tidal forces causes the gravitational effects to be un-conservative. If we tried to calculate the path-integral in this context of GR and QFT together, we would also have to include quantum fluctuations. These fluctuations, combined with the non-conservative fields, make the path-integral terribly divergent.

The problem can be attributed to GRs insistence on "point-like" features of the manifold. This is in contrast to the usual description of events on space-time in QFT, where time and energy are smeared out by Heisenberg's Uncertainty Principle.
\begin{equation}
	\Delta E \Delta t \approx {h \over 2 \pi}
\end{equation}

% , we must take each point with a given precision. According to the uncertainty principle $\Delta E \Delta t \approx {h \over 2 \pi}$ \footnote{The Uncertainty Principle states that for a pair of conjugate variables such as position/momentum and energy/time, it is impossible to have a precisely determined value of each member of the pair at the same time.}, energy and time are related in such a way that the conservation of energy can be violated, although for a very short moment. This allows a temporary change in the amount of energy in a point in the space-time and thereby the creation of particle-antiparticle pairs of virtual particles. This phenomenon is called quantum fluctuations.

\subsubsection{Quantum gravity and the Graviton}
It is possible to describe gravity in the framework of quantum field theory like the other fundamental forces, such that the attractive force of gravity arises due to exchange of virtual gravitons, in the same way as the electromagnetic force arises from exchange of virtual photons. This reproduces general relativity in the classical limit. However, this approach fails at short distances of the order of the Planck length, and high-energies, where a more complete theory of quantum gravity, or a new approach to quantum mechanics, is required, related to the discussion above. Some theory candidates (like string theory and NCG) tries to resolve this by defining a minimum length-scale to smear out quantum fluctuations in hope of making QG renormalizable.
