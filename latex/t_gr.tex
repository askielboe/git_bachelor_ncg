\subsection{General relativity}
General relativity is the theory of gravity developed by Einstein in 1915. While the SM of particle physics is able to account for an amazing variety of electromagnetic, weak and strong interactions, it tells us nothing about gravity. General relativity describes gravity, not by the concept of force, as Newton did, but as a property of the geometry of space and time.

\subsubsection{Space-time}
Given that space and time are related, measurements in space can be related to measurements in time by the speed of light. The resulting space-time is thought of as a smooth and continuous manifold. In QFT the mathematical descriptions depends on fields embedded in the flat Minkowski space-time of special relativity. In GR gravity is modeled as a curvature within space-time, a curvature that changes as mass or energy moves. Thus two bodies have an influence on the space-time curvature in the vicinity of each other. One phenomenon resulting from this is the tidal force.\footnote{Tidal forces are also responsible for tides on Earth, as a result of the varying distance between Earth and the Moon, and the fact that the gravitational attraction between the two bodies varies across their diameter.} You can imagine some kind of bending of the coordinate system around the masses. Therefore GR is incompatible with QFT. The tidal forces causes the gravitational effects to be nonconservative. If we tried to calculate the path-integral in this context of GR and QFT together, we would also have to include quantum fluctuations. These fluctuations, combined with the non-conservative fields, make the path-integral horribly divergent.

The problem can be attributed to GRs insistence on "point-like" features of the manifold. This is in contrast to the usual description of events on space-time in QFT, where position and momentum are smeared out by Heisenberg's Uncertainty Principle
\begin{equation} \label{eq:heisenberg}
	\Delta X \Delta P \geq {\hbar \over 2 \pi}.
\end{equation}

\subsubsection{Quantum gravity and the Graviton}
It is somewhat possible to describe gravity in the framework of quantum field theory, like the other fundamental forces. The attractive force of gravity then arises due to exchange of virtual gravitons, in the same way as the electromagnetic force arises from exchange of virtual photons. This reproduces general relativity in the classical limit. This approach, however, fails at short distances of the order of the Planck length, and high energies, where a more complete theory of quantum gravity, or a new approach to quantum mechanics, is required. Some theoretical candidates (the most well-known being string theory) tries to resolve this by defining a minimum length-scale to smear out quantum fluctuations in hope of making gauge theories involving QG renormalizable.
